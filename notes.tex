\documentclass[fontsize=11pt,a4paper]{scrartcl}
\usepackage[utf8]{inputenc}
\usepackage[english]{babel}
\usepackage{graphicx}
\usepackage{amsmath, amssymb, amsopn}
\usepackage{xcolor} % include before tikz!
\usepackage{tikz}
\usepackage[european]{circuitikz}
\usepackage{nicefrac}
\usepackage{trfsigns} % for \laplace,\Laplace
\usepackage{booktabs}
%\usepackage{framed}
%\usepackage{geometry}
%\geometry{left=0.5cm,right=0.5cm,top=0.5cm,bottom=0.5cm}
\setlength{\parskip}{7pt}
\setlength{\parindent}{0em}
%
%
%
\usepackage[osf]{libertine}
\usepackage{zi4}
\usepackage[libertine,cmbraces]{newtxmath}
%
%
%
%
\DeclareMathOperator{\grad}{grad}
\DeclareMathOperator{\Div}{div}
\DeclareMathOperator{\rot}{rot}
\DeclareMathOperator{\sinc}{sinc}
%
%
%
%
\usepackage{listings}
\usepackage{bytefield}
%
%
%
%
\begin{document}
\clearpage
\begingroup
	\pagestyle{empty}
	\begin{center}
		\LARGE{\textbf{Notes}}
	\end{center}
	\hfill
	\tableofcontents
	\clearpage
\endgroup
\newpage
\setcounter{page}{1}
%
%
%
%
\lstset{numbers=left,
	frame=single,
	numberstyle=\tiny,
	basicstyle=\footnotesize,
	showstringspaces=false,
	%keywordstyle=\color{blue},
	%commentstyle=\em\color{gray},
	tabsize=3,
	numbersep=5pt,
	%morecomment=[s][\color{blue}]{<<<}{>>>},
	%morekeywords={float3,float4,__device__,__global__,__shared__,__constant__,threadIdx,blockIdx,blockDim,gridDim,\_\_syncthreads}}
}
%
%
%
\section{Verilog}
\subsection{Synthesis}
\lstset{language=verilog}
Listing \ref{lst:basic_structure} shows the basic structure of a Verilog module. Identifiers are case-sensitive.
\begin{figure}[htb]
\begin{lstlisting}
// Single line comment
/* Comment spanning
   several lines */

module some_module (
	// Ports
	input [wire/reg] in1,
	input in2,
	output [wire/reg] out
);
	// Internal signals
	wire signal1, signal2;

	// Vector, bus (32 bits wide)
	wire [31:0] bus;

	// Examples for accessing part of a vector:
	// bus[0], bus[7:0]

	// Local parameter, cannot be changed
	localparam PARAM = 42;

endmodule
\end{lstlisting}
\caption{Basic structure of a Verilog module}
\label{lst:basic_structure}
\end{figure}

Signals can take the values \lstinline!0! (logic $0$), \lstinline!1! (logic $1$), \lstinline!x! (undefined, uninitialized) and \lstinline!z! (high impedance).

Integer constants: \lstinline!<bits>'<base><literal>! (leaving out \lstinline!<bits>! defaults to a width of at least 32 bits, underscores in \lstinline!<literal>! are ignored). Valid values for \lstinline!<base>!: \lstinline!b! (binary), \lstinline!d! (decimal), \lstinline!o! (octal) and \lstinline!h! (hexadecimal). Examples: \lstinline!4'b01_01!, \lstinline!8'hab!, \lstinline!10!.

Instantiating modules or primitives connecting ports by sequence:
\begin{lstlisting}
some_module instance_name (
	// Inputs and outputs in the same order
	// as the appear in the declaration of some_module
	in1,
	in2,
	out
);
\end{lstlisting}

Connecting the ports by name:
\begin{lstlisting}
some_module instance_name (
	.IN1(in1),
	.IN2(in2),
	.OUT(out)
);
\end{lstlisting}

Dataflow modeling (combinatorial circuit)
\begin{lstlisting}
module my_xor (
	input a,
	input b,
	output y
);

	assign y = a ^ b;

endmodule
\end{lstlisting}

Operators
\begin{table}[htb]
	\centering
	\begin{tabular}{lll}
	\toprule
		\textbf{Operator} & \textbf{Type} & \textbf{Meaning} \\
	\midrule
		\lstinline!&! & Bitwise & AND \\
		\lstinline!|! & Bitwise & OR \\
		\lstinline!^! & Bitwise & Exclusive OR \\
		\lstinline!~! & Bitwise & One's complement \\
		\lstinline!{a, b, c}! & Other & Concatenate wires or vectores \\
		... & ... & ...\\
	\bottomrule
	\end{tabular}
	\caption{Verilog operators}
	\label{tab:operators}
\end{table}

Behavioral modeling using \lstinline!always! blocks:
\begin{lstlisting}
// Synchronous logic
// Edge triggered
always @(posedge clock, negedge reset) begin
	if (reset)
		counter <= 0;
	else
		counter <= counter + 1;
end

// Asynchronous logic
// Sensitive to all signals in block, level triggered
always @* begin
	// ...
end
\end{lstlisting}

Non-blocking assignment: \lstinline!signal <= /* ... */!

Blocking assignment: \lstinline!signal = /* ... */!

Switch case
\begin{lstlisting}
case (signal)
	value1: /* instruction / block */
	value2: /* instruction / block */
	// ...
	default: /* instruction / block */
endcase
\end{lstlisting}

\lstinline!casez! treats \lstinline!z!, \lstinline!casex! \lstinline!x! and \lstinline!z! as don't care values.

Use default values to avoid inference of latches/ registers.

Modeling of memory with arrays:
\begin{lstlisting}
// 256 bytes of memory
reg [7:0] mem [0:255];
//    ^         ^
//    |         |
//    |         -- Number of cells
//    ------------ Memory width
\end{lstlisting}

Tristate ports: \lstinline!inout!, Pull-down: \lstinline!tri0!, Pull-up: \lstinline!tri1!, Wired-AND: \lstinline!wand! or \lstinline!triand!, Wired-OR: \lstinline!wor! or \lstinline!trior!

Declaration and instantiation of modules with parameters:
\begin{lstlisting}
module some_module #(
	// Parameters with default values
	parameter PARAM1 = 1,
	parameter PARAM2 = 2
)(
	input in,
	output out
);

	/* ... */

endmodule

// Instantiation of the module above
some_module #(
	.PARAM1(param1),
	.PARAM2(param2)
) instance_name (
	/* ... */
);
\end{lstlisting}

Functions: only combinational circuits, no registers, delays and non-blocking assignments (are defined in the module in which they are used, can call other functions).
\begin{lstlisting}
function my_xor;
	input a, b;

	begin
		my_xor = a ^ b;
	end
endfunction

assign y = my_xor(c, d);
\end{lstlisting}

Tasks: no return value, can have \lstinline!ouput! and \lstinline!inout! ports, can have delays
\begin{lstlisting}
task my_inverter;
	// FIXME
\end{lstlisting}

\lstinline!generate! Blocks:
\begin{lstlisting}
	// FIXME
\end{lstlisting}

Compiler directives: FIXME
%
%
%
%
\subsection{Simulation}
Register data types: \lstinline!integer!, \lstinline!real!, \lstinline!time!

Initial blocks:
\begin{lstlisting}
module testbench;
	initial begin
		$display("Hello world!);
		$finish;
	end
endmodule
\end{lstlisting}

Delays: \lstinline!#10;!

System tasks: \lstinline!$display("Format string", /* ... */)!

\lstinline!$monitor("Format string", /* ... */)!: automatically generates output when one of the values changes

VCD (Value Change Dump) files:
\begin{lstlisting}
$dumpfile("traces.vcd")
$dumpvars(0, testbench)
\end{lstlisting}
\end{document}
