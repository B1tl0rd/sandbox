\documentclass[fontsize=11pt,a4paper]{scrartcl}
\usepackage[utf8]{inputenc}
\usepackage[english]{babel}
\usepackage{graphicx}
\usepackage{amsmath, amssymb, amsopn}
\usepackage{xcolor} % include before tikz!
\usepackage{tikz}
\usepackage[european]{circuitikz}
\usepackage{nicefrac}
\usepackage{trfsigns} % for \laplace,\Laplace
\usepackage{booktabs}
%\usepackage{framed}
%\usepackage{geometry}
%\geometry{left=0.5cm,right=0.5cm,top=0.5cm,bottom=0.5cm}
\setlength{\parskip}{7pt}
\setlength{\parindent}{0em}
%
%
%
\usepackage[osf]{libertine}
\usepackage{zi4}
\usepackage[libertine,cmbraces]{newtxmath}
%
%
%
%
\DeclareMathOperator{\grad}{grad}
\DeclareMathOperator{\Div}{div}
\DeclareMathOperator{\rot}{rot}
\DeclareMathOperator{\sinc}{sinc}
\DeclareMathOperator{\curl}{curl}
\DeclareMathOperator{\diag}{diag}
%
%
%
%
\usepackage{listings}
%\usepackage{bytefield}
%
%
%
%
\usepackage{datetime} % for \currenttime
\usepackage{braket}
\usepackage{upgreek}
%
%
%
%
\begin{document}
\clearpage
\begingroup
	\pagestyle{empty}
	\begin{center}
		\LARGE{\textbf{Notes}}

		\small{Compiled on {\today} at \currenttime}
	\end{center}
	\hfill
	\tableofcontents
	\clearpage
\endgroup
\newpage
\setcounter{page}{1}
%
%
%
%
\lstset{numbers=left,
	frame=single,
	numberstyle=\tiny,
	basicstyle=\footnotesize,
	showstringspaces=false,
	%keywordstyle=\color{blue},
	%commentstyle=\em\color{gray},
	tabsize=3,
	numbersep=5pt,
	%morecomment=[s][\color{blue}]{<<<}{>>>},
	%morekeywords={float3,float4,__device__,__global__,__shared__,__constant__,threadIdx,blockIdx,blockDim,gridDim,\_\_syncthreads}}
}
%
%
%
\section{Verilog}
\subsection{Synthesis}
\lstset{language=verilog}
Listing \ref{lst:basic_structure} shows the basic structure of a Verilog module. Identifiers are case-sensitive.
\begin{figure}[htb]
\begin{lstlisting}
// Single line comment
/* Comment spanning
   several lines */

module some_module (
	// Ports
	input [wire/reg] in1,
	input in2,
	output [wire/reg] out
);
	// Internal signals
	wire signal1, signal2;

	// Vector, bus (32 bits wide)
	wire [31:0] bus;

	// Examples of accessing part of a vector:
	// bus[0], bus[7:0]

	// Local parameter, cannot be changed
	localparam PARAM = 42;

endmodule
\end{lstlisting}
\caption{Basic structure of a Verilog module}
\label{lst:basic_structure}
\end{figure}

Signals can take the values \lstinline!0! (logic $0$), \lstinline!1! (logic $1$), \lstinline!x! (undefined, uninitialized) and \lstinline!z! (high impedance).

Integer constants: \lstinline!<bits>'<base><literal>! (leaving out \lstinline!<bits>! defaults to a width of at least 32 bits, underscores in \lstinline!<literal>! are ignored). Valid values for \lstinline!<base>!: \lstinline!b! (binary), \lstinline!d! (decimal), \lstinline!o! (octal) and \lstinline!h! (hexadecimal). Examples: \lstinline!4'b01_01!, \lstinline!8'hab!, \lstinline!10!.

Instantiating modules or primitives connecting ports by sequence:
\begin{lstlisting}
some_module instance_name (
	// Inputs and outputs in the same order
	// as the appear in the declaration of some_module
	in1,
	in2,
	out
);
\end{lstlisting}

Connecting the ports by name:
\begin{lstlisting}
some_module instance_name (
	.IN1(in1),
	.IN2(in2),
	.OUT(out)
);
\end{lstlisting}

Dataflow modeling (combinatorial circuit)
\begin{lstlisting}
module my_xor (
	input a,
	input b,
	output y
);

	assign y = a ^ b;

endmodule
\end{lstlisting}

Operators
\begin{table}[htb]
	\centering
	\begin{tabular}{lll}
	\toprule
		\textbf{Operator} & \textbf{Type} & \textbf{Meaning} \\
	\midrule
		\lstinline!&! & Bitwise & AND \\
		\lstinline!|! & Bitwise & OR \\
		\lstinline!^! & Bitwise & Exclusive OR \\
		\lstinline!~! & Bitwise & One's complement \\
		\lstinline!{a, b, c}! & Other & Concatenate wires or vectores \\
		... & ... & ...\\
	\bottomrule
	\end{tabular}
	\caption{Verilog operators}
	\label{tab:operators}
\end{table}

Behavioral modeling using \lstinline!always! blocks:
\begin{lstlisting}
// Synchronous logic
// Edge triggered
always @(posedge clock, negedge reset) begin
	if (reset)
		counter <= 0;
	else
		counter <= counter + 1;
end

// Asynchronous logic
// Sensitive to all signals in block, level triggered
always @* begin
	// ...
end
\end{lstlisting}

Non-blocking assignment: \lstinline!signal <= /* ... */!

Blocking assignment: \lstinline!signal = /* ... */!

Switch case
\begin{lstlisting}
case (signal)
	value1: /* instruction / block */
	value2: /* instruction / block */
	// ...
	default: /* instruction / block */
endcase
\end{lstlisting}

\lstinline!casez! treats \lstinline!z!, \lstinline!casex! \lstinline!x! and \lstinline!z! as don't care values.

Use default values to avoid inference of latches/ registers.

Modeling of memory with arrays:
\begin{lstlisting}
// 256 bytes of memory
reg [7:0] mem [0:255];
//    ^         ^
//    |         |
//    |         -- Number of cells
//    ------------ Memory width
\end{lstlisting}

Tristate ports: \lstinline!inout!, Pull-down: \lstinline!tri0!, Pull-up: \lstinline!tri1!, Wired-AND: \lstinline!wand! or \lstinline!triand!, Wired-OR: \lstinline!wor! or \lstinline!trior!

Declaration and instantiation of modules with parameters:
\begin{lstlisting}
module some_module #(
	// Parameters with default values
	parameter PARAM1 = 1,
	parameter PARAM2 = 2
)(
	input in,
	output out
);

	/* ... */

endmodule

// Instantiation of the module above
some_module #(
	.PARAM1(param1),
	.PARAM2(param2)
) instance_name (
	/* ... */
);
\end{lstlisting}

Functions: only combinational circuits, no registers, delays and non-blocking assignments (are defined in the module in which they are used, can call other functions).
\begin{lstlisting}
function my_xor;
	input a, b;

	begin
		my_xor = a ^ b;
	end
endfunction

assign y = my_xor(c, d);
\end{lstlisting}

Tasks: no return value, can have \lstinline!ouput! and \lstinline!inout! ports, can have delays
\begin{lstlisting}
task my_inverter;
	// FIXME
\end{lstlisting}

\lstinline!generate! Blocks:
\begin{lstlisting}
	// FIXME
\end{lstlisting}

Compiler directives: FIXME
%
%
%
%
\subsection{Simulation}
Register data types: \lstinline!integer!, \lstinline!real!, \lstinline!time!

Initial blocks:
\begin{lstlisting}
module testbench;
	initial begin
		$display("Hello world!);
		$finish;
	end
endmodule
\end{lstlisting}

Delays: \lstinline!#10;!

System tasks: \lstinline!$display("Format string", /* ... */)!

\lstinline!$monitor("Format string", /* ... */)!: automatically generates output if one of the values changes

VCD (Value Change Dump) files:
\begin{lstlisting}
$dumpfile("traces.vcd")
$dumpvars(0, testbench)
\end{lstlisting}
%
%
%
%
%
\section{VHDL}
\lstset{language=vhdl,
        morecomment=[s]{/*}{*/},
	morekeywords={unaffected,parameter}
}
\subsection{Lexical Elements}
Types of \emph{comments} (can't be nested):
\begin{lstlisting}
-- Single line comment
/* Comment spanning multiple
   lines (VHDL-2008 only) */
\end{lstlisting}

Valid \emph{identifiers} consist of the characters \lstinline!'a'!--\lstinline!'z'!, \lstinline!'A'!--\lstinline!'Z'!, \lstinline!'0'!--\lstinline!'9'!, \lstinline!'_'! and may only start with an alphabetic letter. Identifiers with more than one consecutive underscore or identifiers that end with an underscore are illegal. Furthermore, identifiers are case insensitive in VHDL.

Examples of \emph{numeric literals}:
\begin{lstlisting}
-- Integer literals (base 10)
1 42 10e5 10E6
1_234           -- Identical to 1234, underscores get ignored

-- Real literals (base 10, always contain the decimal point '.' character)
3.1415927 1.6022e-19 8.8542E-12

-- Other bases
2#10101010#   -- Binary integer
8#777#        -- Octal integer
16#aa# 16#FF# -- Hexadecimal integer
2#1.101#      -- Binary real
\end{lstlisting}

Examples of \emph{character and string literals}:
\begin{lstlisting}
-- Characters (ISO 8859 Latin-1 8 bit character set)
'a' 'b' 'X' '_'
'''             -- Single quote

-- Strings (may not be separated by linebreaks)
"Hello world!"
""""           -- Double quote
""             -- Empty string
\end{lstlisting}

\newpage
\emph{Bit string literals} are enclosed in double quotes with a prefix that indicates the size of the string in bits (\textsc{VHDL-2008} only) and the base. Examples:
\begin{lstlisting}
b"101010" B"1111_0000" -- Binary (underscores are ignored)
"11001100"             -- Equivalent to b"11001100"
o"777" O"777"          -- Octal
x"aabb" X"FFFF"        -- Hexadecimal

-- VHDL-2008 only
8b"1111_0000" -- 8 bit long binary number
d"42" D"23"   -- Decimal number, size can't be specified

X"0X1"       -- Equivalent to B"0000XXXX1111"
B"1111_ZZ00" -- Equivalent to B"1111ZZ00"

UB"" UO"" UX"" -- Unsigned binary, octal, hexadecimal
SB"" SO"" SX"" -- Signed     "       "         "
\end{lstlisting}
%
%
%
%
\subsection{Fundamental Data Types}
Declaring \emph{constants} (one of several types of objects, namely constants, variables, signals and files):\\ \lstinline!constant identifier {, ...}: type_name [:= initial_value];!

Examples:
\begin{lstlisting}
constant pi : real    := 3.1415927;
constant n  : natural := 42;
\end{lstlisting}

Declaring \emph{variables}:\\ \lstinline!variable identifier {, ...}: type_name [:= initial_value];!

Examples:
\begin{lstlisting}
variable count : integer := 0;
variable x     : natural;
\end{lstlisting}

\emph{Assignments} to variables are non-blocking (in contrast to assignments to signals), i.e. the value of the variable gets updated immediately:\\ \lstinline![label_name:] identifier := value;!

Examples:
\begin{lstlisting}
x     := 1;
count := count + 1;
\end{lstlisting}

New \emph{types} can be declared using\\ \lstinline!type identifier is definition;!.

Declaring new \emph{integer} types:\\ \lstinline!type identifier is range bound1 (to|downto) bound2;!

Example:
\begin{lstlisting}
type day_of_week is range 0 to 6;

constant JAN : integer := 1;
constant DEC : integer := 12;
type month is range JAN to DEC;  -- Bounds for range can be evaluated
                                 -- at compile time
\end{lstlisting}
Operations that are defined for integer types include \lstinline!+!, \lstinline !-!, \lstinline!*!, \lstinline!/!, \lstinline!mod!, \lstinline!rem!, \lstinline!abs! and \lstinline!**!.

New \emph{floating point} types are declared exactly as their integer counterparts, except the bounds for their range have to evaluate to real numbers. If the default value for floating point variables is left out, they default to the lowest possible value.

\emph{Physical} types include a physical unit and can be declared as follows:
\begin{lstlisting}
type identifier is range bound1 (to|downto) bound2
	units
		identifier;
		{identifier2 = (integer_literal|real_literal) unit;}
	end units [identifier];
\end{lstlisting}

Example:
\begin{lstlisting}
type voltage is range 0 to 1e9
	units
		volt;              -- Base unit
		kvolt = 1000 volt;
	end units voltage;
\end{lstlisting}
VHDL includes a predefined physical type \lstinline!time!, that defines the following unit prefixes: \lstinline!fs!, \lstinline!ps!, \lstinline!ns!, \lstinline!us!, \lstinline!ms!, \lstinline!sec!, \lstinline!min! and \lstinline!hr!.

Declaring \emph{enumeration} types:\\ \lstinline!type identifier is ( (identifier|character_literal) {, ...} );!

Examples:
\begin{lstlisting}
type fsm_state is (reset, state1, state2);
variable state : fsm_state := reset;

type hexdigit is ('0','1','2','3','4','5','6','7','8','9',
                  'a','b','c','d','e','f'); -- Lowercase hex digits
variable x : hexdigit := 'a';

-- Predefined character enumeration type
variable c : character := 'x';
\end{lstlisting}

The predefined \lstinline!boolean! enumeration type can take the values \lstinline!true! and \lstinline!false!. Valid expressions including booleans may use the operators \lstinline!=! (is equal), \lstinline!/=! (is not equal), \lstinline!and!, \lstinline!or!, \lstinline!nand!, \lstinline!nor!, \lstinline!not!, \lstinline!xor! and \lstinline!xnor!.

Predefined \lstinline!bit! type: \lstinline!type bit is ('0', '1');! (overloads \lstinline!'0'! and \lstinline!'1'! character literals).

Declaring \emph{subtypes}:\\ \lstinline!subtype identifier is type_name range bound1 (to|downto) bound2;!

Examples:
\begin{lstlisting}
subtype uint8 is integer range 0 to 255;

variable b : uint8 := 255;

-- Predefined subtype for delays >=0 fs
variable delay : delay_length;
\end{lstlisting}

The \emph{type qualification} operator \lstinline!typename'(expression)! can be used to distinguish between overloaded enumeration literals:
\begin{lstlisting}
type fsm1_state is (reset, a1, a2);
type fsm2_state is (reset, b1, b2);

fsm1_state'(reset) -- Type of reset literal: fsm1_state
fsm2_state'(reset) -- Type of reset literal: fsm2_state
\end{lstlisting}

Examples of \emph{type conversions}:
\begin{lstlisting}
real(42)         -- Integer to real
integer(2.71828) -- Real to integer
\end{lstlisting}

List of \emph{attributes} applicable to scalar data types (\lstinline!T! denotes a type name):

%\centering
\begin{tabular}{ll}
\toprule
	\textbf{Attribute} & \textbf{Description}\\
\midrule
	\lstinline!T'ascending! & \lstinline!True!: ascending range, \lstinline!false!: descending range \\
	\lstinline!T'low! & Lowest possible value\\
	\lstinline!T'high! & Highest possible value\\
	\lstinline!T'left! & Leftmost element\\
	\lstinline!T'right! & Rightmost element\\
	\lstinline!T'image(value)! & Converts \lstinline!value! to string representation\\
	\lstinline!T'value(string)! & Converts \lstinline!string! to value\\
\bottomrule
\end{tabular}


List of additional attributes applicable to discrete and physical types (\lstinline!T! denotes a type name):

%\centering
\begin{tabular}{ll}
\toprule
	\textbf{Attribute} & \textbf{Description}\\
\midrule
	\lstinline!T'pos(value)! & Index of \lstinline!value!\\
	\lstinline!T'val(j)! & Value at index \lstinline!j!\\
	\lstinline!T'succ(value)! & \lstinline!Value! incremented by one\\
	\lstinline!T'pred(value)! & \lstinline!Value! decremented by one\\
	\lstinline!T'leftof(value)! & Element to the left of \lstinline!value!\\
	\lstinline!T'rightof(value)! & Element to the right of \lstinline!value!\\
\bottomrule
\end{tabular}
%
%
%
%
\subsection{Sequential Control Flow Constructs}
Syntax of \emph{if} statements:
\begin{lstlisting}
[label_name:]
-- VHDL-2008 only: implicit conversion of 'expression' to boolean
if expression then
	{...}
{elsif expression then
	{...}}
[else                   -- Default branch
	{...}]
end if [label_name];
\end{lstlisting}

Example:
\begin{lstlisting}
entity bincnt is
	port (clk : in std_logic;
	      led1, led2, led3, led4, led5 : out std_logic);
end bincnt;

architecture behavioral of bincnt is
	signal cnt : unsigned (4 downto 0) := (others => '0');
begin
	process (clk)
		variable delay : unsigned (23 downto 0) := (others => '0');
	begin
		if rising_edge(clk) then
			if delay = 2_999_999 then
				cnt <= cnt + 1;
				delay := x"000000";
			else
				delay := delay + 1;
			end if;
		end if;
	end process;

	(led5,led4,led3,led2,led1) <= std_logic_vector(cnt);
end behavioral;
\end{lstlisting}

Syntax of \emph{conditional variable assignments} (VHDL-2008 only):
\begin{lstlisting}
[label_name:]
variable := value when expression
	{else value when expression}
	[else expression];            -- Default branch
\end{lstlisting}

Syntax of \emph{case} statements:
\begin{lstlisting}
[label_name:]
case expression is
	when (expression
	      | bound1 (to|downto) bound2 -- Discrete range
	      | subtype_name              -- Name of subtype
	      | others)                   -- Default branch
	=> {...}
end case [label_name];
\end{lstlisting}
Multiple alternatives in a \lstinline!when! statement can be separated with the ``\lstinline!|!'' character, e.g.
\begin{lstlisting}
case x is
	when 1|2    =>
		y <= 1;
	when 3      =>
		y <= 2;
	when others =>
		y <= 3;
end case;
\end{lstlisting}

Another example of a case statement:
\begin{lstlisting}
entity mux4 is
	port (a0, a1, a2, a3 : in  bit;
	      sel0, sel1     : in  bit;
	      y              : out bit);
	constant T_PD : delay_length := 4.5 ns;
end entity mux4;

architecture behav of mux4 is
	signal sel : bit_vector(0 to 1);
begin
	sel <= (sel0, sel1);

	process (sel, a0, a1, a2, a3) is
	begin
		case sel is
			when "00" => y <= a0 after T_PD;
			when "01" => y <= a1 after T_PD;
			when "10" => y <= a2 after T_PD;
			when "11" => y <= a3 after T_PD;
		end case;
	end process;
end architecture behav;
\end{lstlisting}

Syntax of \emph{selected variable assignments} (VHDL-2008 only):
\begin{lstlisting}
[label_name:]
with expression select
variable := {value when expression,}
	value when expression;
\end{lstlisting}

\emph{Null} statements (no operation): \lstinline![label_name:] null;!.

General syntax of \emph{loops}:
\begin{lstlisting}
[label_name:]
[while expression                               -- Boolean expression
 | for identifier in bound1 (to|downto) bound2] -- Discrete range
loop                                            -- Loops forever without
                                                -- while or for
	{...}
end loop [label_name];
\end{lstlisting}

Loops can be aborted with the \lstinline!exit! statement: \lstinline![label_name:] exit [label_loop] [when expression];!, to continue with the next iteration of the loop one can use the \lstinline!next! statement:\\ \lstinline![label_name:] next [label_loop] [when expression];!.

Example of a loop without \lstinline!while! or \lstinline!for!:
\begin{lstlisting}
entity counter is
	port (clk   : in bit;
	      count : out natural);
end entity counter;

architecture behav of counter is
begin
	process is
		variable c : natural := 15;
	begin
		count <= c;
		loop
			wait until clk = '1';
			c := c - 1 when c /= 0
			     else 15;
			count <= c;
		end loop;
	end process;
end architecture behav;
\end{lstlisting}

Example of a \emph{for loop}:
\begin{lstlisting}
entity average is
	port (clk  : in bit;
	      din  : in real;
	      dout : out real);
end entity average;

architecture behav of average is
	constant N : integer := 4;
begin
	process is
		variable sum : real;
	begin
		loop
			sum := 0.0;
			for i in 0 to N-1 loop
				wait until clk = '1';
				sum := sum + din;
			end loop;

			dout <= sum / real(N);
		end loop;
	end process;
end architecture behav;
\end{lstlisting}

Syntax of \emph{assertions} (for simulation, synthesis or formal verification):
\begin{lstlisting}
[label_name:] assert expression
         [report expression] [severity level];

-- Valid values for level:
-- note, warning, error, failure
\end{lstlisting}

\emph{Report} statements are equivalent to unconditional assertions:\\ \lstinline![label_name:] report expression [severity level];!.
%
%
%
%
\subsection{Composite Data Types}
Declaring new \emph{array} types:
\begin{lstlisting}
type identifier is array
	(discrete_subtype                                   -- Index is a discrete
	                                                    -- subtype
	 | discrete_subtype range bound1 (to|downto) bound2 -- Index is a discrete
	                                                    -- subtype with
	                                                    -- constrained range
	 | bound1 (to|downto) bound2                        -- Index range is
	)                                                   -- specified directly
	 {, ...}                                            -- Multidimensional
	                                                    -- array
	) of subtype;
\end{lstlisting}

Example of a one-dimensional array:
\begin{lstlisting}
entity reg is
	port (din         : in bit_vector (31 downto 0);
	      read, write : in integer range 0 to 15;
	      wen         : in bit;
	      dout        : out bit_vector (31 downto 0));
end entity reg;

architecture behav of reg is
begin
	process (din, read, write, wen) is
		type mem is array (0 to 15) of bit_vector (31 downto 0);
		variable memory : mem;
	begin
		if wen = '1' then
			memory(write) := din;
		end if;
		dout <= memory(read);
	end process;
end architecture behav;
\end{lstlisting}

Array literals can be constructed out of scalar data types using \emph{array aggregates}:
\begin{lstlisting}
([expression                   -- Named association: indices explicitly
                               --                    specified
  | discrete_subtype
  | bound1 (to|downto) bound2
  | others => ]                -- Wildcard for any indices that haven't been
                               -- specified previously (must be last)
  expression                   -- Positional association is used
                               -- if the [... => ] part is left out
{, ...})
\end{lstlisting}

Example:
\begin{lstlisting}
type narray is array (0 to 3) of natural;

variable n1 : narray := (1, 2, 3, 4);     -- Positional association
variable n2 : narray := (0 => 1, 1 => 2,  -- Named association
                         2 => 3, 3 => 4);
\end{lstlisting}

Multiple indices can be separated with the ``\lstinline!|!'' character:
\begin{lstlisting}
type rarray is array (1 to 10) of real;

variable r : rarray := (1|2|4 => 3.14, 5 to 7 => 2.718, others => 0.0);
\end{lstlisting}

Array aggregates can also be used as the target in both variable and signal assignments:
\begin{lstlisting}
[label_name:] (identifier|array_aggregate) := expression;
[label_name:] (identifier|array_aggregate) <= expression;

-- Example:
variable a, b, c, d : bit;
variable x          : bit_vector (3 downto 0);

(a, b, c, d) := x;
\end{lstlisting}

In addition to just scalar types, VHDL-2008 also allows the use of sub-arrays in the specification of array aggregates.

Attributes that can be used in conjunction with array types are (\lstinline!A! denotes either an array type or object, the index \lstinline!j=1,2,...! which dimension the attribute shall refer to. If the parentheses are left out, \lstinline!j! defaults to one.):

\begin{tabular}{ll}
\toprule
	\textbf{Attribute} & \textbf{Description}\\
\midrule
	\lstinline!A'left(j)! & Leftmost index\\
	\lstinline!A'right(j)! & Rightmost index\\
	\lstinline!A'low(j)! & Lowest index\\
	\lstinline!A'high(j)! & Highest index\\
	\lstinline!A'range(j)! & Index range\\
	\lstinline!A'reverse_range(j)! & Reversed index range\\
	\lstinline!A'length(j)! & Length of index range\\
	\lstinline!A'ascending(j)! & \lstinline!True!: ascending range, \lstinline!false!: descending range\\
	\lstinline!A'element! & Subtype of the elements in \lstinline!A! (VHDL-2008 only)\\
\bottomrule
\end{tabular}

Array types that leave the index range(s) unspecified are called \emph{unconstrained} array types and can be declared in the following way:
\begin{lstlisting}
type identifier is array
	(type_name range <>
	{, ...})          -- Multidimensional array
) of subtype;
\end{lstlisting}

Example:
\begin{lstlisting}
package array_type is
	type int_array is array (natural range <>) of integer;
end package array_type;

use work.array_type.all;

entity max is
	port (din  : in int_array;
	      dout : out integer);
end entity max;

architecture behav of max is
begin
	process (din) is
		variable max : integer;
	begin
		assert din'length /= 0
			report "Array has zero length!"
			severity failure;

		max := din(din'left);
		for i in din'range loop
			if din(i) > max then
				max := din(i);
			end if;
			dout <= max;
		end loop;
	end process;
end architecture behav;
\end{lstlisting}

Predefined unconstrained array types include \lstinline!string! and \lstinline!bit_vector!, VHDL-2008 additionally provides \lstinline!boolean_vector!, \lstinline!integer_vector!, \lstinline!real_vector! and \lstinline!time_vector!.

Parts of an array can be referenced using \emph{array slices}:\\ \lstinline!identifier (bound1 (to|downto) bound2)!.

Declaring new \emph{record} types:
\begin{lstlisting}
type identifier is record
	identifier {, ...} : type_name;
	{...}
end record [identifier];
\end{lstlisting}

Single members in a record type can be referred to using \emph{selected names}: \lstinline!identifier.record_member!.

Example:
\begin{lstlisting}
type coordinate is record
	x, y, z : real;
end record coordinate;

variable v1, v2 : coordinate;

v1.x = 1.0; v1.y = 2.0; v1.z = 3.0;
v2 = v1;
\end{lstlisting}

\emph{Record aggregates} can be constructed similarly to array aggregates:
\begin{lstlisting}
([expression     -- Named association: record members explicitly specified,
                 -- multiple choices may be separated with the '|' character
  | others => ]  -- Wildcard for any members that haven't been
                 -- specified previously (must be last)
  expression     -- Positional association is used
                 -- if the [... => ] part is left out
{, ...})
\end{lstlisting}

Example:
\begin{lstlisting}
v1 = (1.0, 2.0, 3.0);                -- Positional association
v1 = (x => 1.0, y => 2.0, z => 3.0); -- Named association
v2 = (others => 0.0);
\end{lstlisting}
%
%
%
%
\newpage
\subsection{Entities and Architectures, Behavioral and Structural Modeling}
Syntax of \emph{entity} declarations:
\begin{lstlisting}
entity identifier is
	[generic (
		identifier {, ...} : typename
			[:= default_value]
		-- Generic type (VHDL-2008 only)
		| type generic_type_identifier
		{; ...}
	);]

	[port (
		identifier {, ...} : (in|out|buffer|inout) typename
			[:= default_value]
		-- Generic type (VHDL-2008 only)
		| generic_type_identifier ...
		{; ...}
	);]

	-- Declarations common to all implementations
	{...}

[begin
	-- Concurrent assertion statements
	-- Passive concurrent procedure calls
	-- Passive processes
	{...}
]
end [entity] [identifier];
\end{lstlisting}

Procedure calls and processes are \emph{passive} if they don't include any signal assignments.

\lstinline!In! and \lstinline!out! ports are unidirectional, \lstinline!inout! ports are bidirectional and \lstinline!buffer! ports are \lstinline!out! ports that can be read internally (VHDL-2008 also allows reading \lstinline!out! ports).

Syntax of \emph{architectures}:
\begin{lstlisting}
architecture identifier of entity_name is
	-- Declarations
	{...}
begin
	-- Concurrent statements
	{...}
end [architecture] [identifier];
\end{lstlisting}

\emph{Signals} (internal connections within an architecture) are declared as follows:\\ \lstinline!signal identifier {, ...} : type_name [:= initial_value];!.

Syntax of \emph{signal assignments}:
\begin{lstlisting}
-- Default delay mechanism: inertial
[label_name:]
identifier <=
	[transport
	 | [reject time_expression] inertial]
	value [after time_expression] {, ...};
	| unaffected;
\end{lstlisting}

If one assigns the \lstinline!unaffected! value to a signal, its value remains unchanged (VHDL-2008: \lstinline!unaffected! may also be used in sequential signal assignments). Not specifying any delay is equivalent to specifying a delay of \lstinline!0 fs! (\emph{delta delay}).

Syntax of \emph{conditional signal assignments} (VHDL-2008 only, replace with \lstinline!if! in VHDL$<$2008):
\begin{lstlisting}
[label_name:]
identifier <=
	[transport                      -- Delay mechanism
	 | [reject time_expression] inertial]
	value [after time_expression] {, ...};
	| unaffected when expression
	{else ... when expression}
	[else ...]; -- Default branch
\end{lstlisting}

Syntax of \emph{selected signal assignments} (VHDL-2008 only, replace with \lstinline!case! in VHDL$<$2008):
\begin{lstlisting}
[label_name:]
with expression select
identifier <=
	{... when ...,}
	[transport                      -- Delay mechanism
	 | [reject time_expression] inertial]
	value [after time_expression] {, ...};
	| unaffected when expression;
\end{lstlisting}

List of attributes applicable to signals (\lstinline!S! denotes a signal, \lstinline!T! an expression of type \lstinline!time!):

%%%%%% !!!!!FIXME!!!!!!
\begin{tabular}{ll}
\toprule
	\textbf{Attribute} & \textbf{Description}\\
\midrule
	\lstinline!S'active! & \lstinline!True! if signal \lstinline!S! got updated with a new value in the current\\ simulation cycle (\emph{transaction}), \lstinline!false! otherwise.\\
	\lstinline!S'delayed(T)! & Delays signal \lstinline!S! by a time interval \lstinline!T!.\\
	\lstinline!S'event!& \lstinline!True! if the new value of a signal \lstinline!S! is unequal to its old value (\emph{event}),\\ \lstinline!false! otherwise.\\
	\lstinline!S'last_active! & Time since last transaction on signal \lstinline!S!.\\
	\lstinline!S'last_event! & Time since last event on signal \lstinline!S!.\\
	\lstinline!S'last_value! & Last value of signal \lstinline!S! before current event.\\
	\lstinline!S'quiet! & \lstinline!True! if no events occured on signal \lstinline!S! for a time span \lstinline!T!,\\ \lstinline!false! otherwise (return type: \lstinline!signal!).\\
	\lstinline!S'stable(T)! & \lstinline!True! if no events where scheduled on signal \lstinline!S! for a time period \lstinline!T!,\\ \lstinline!false! otherwise (return type: \lstinline!signal!).\\
	\lstinline!S'transaction! & Toggles between \lstinline!'0'! and \lstinline!'1'! each time a transaction occurs\\ on signal \lstinline!S! (return type: \lstinline!signal!).\\
\bottomrule
\end{tabular}

Syntax of \emph{wait statements}:
\begin{lstlisting}
[label_name:]
-- Type of identifier: signal
wait [on identifier {, ...}] -- Sensitivity list
	[until expression]        -- Condition clause
	[for time_expression];    -- Timeout
\end{lstlisting}

Combinational logic can be modeled using \lstinline!wait on! statements. The execution of the process in the following example resumes every time an event occurs on any signal in its \emph{sensitivity list}:
\begin{lstlisting}
my_or: process
begin
	y <= x1 or x2;
	wait on x1, x2;
end process;

-- Equivalent to:
my_or2: process (x1, x2)
begin
	y <= x1 or x2;
end process;
\end{lstlisting}

In contrast, \lstinline!wait until! statements suspend a process until the condition becomes true once again. When combining both aforementioned forms of wait statements, events on signals in the sensitivity list have a higher priority than any conditions. A timeout for maximum suspension of a process may be specified using the \lstinline!for! statement. In simulation, a process can be stopped using a single \lstinline!wait;! instruction.

Syntax of \emph{process statements}:
\begin{lstlisting}
-- Type of identifier: signal
[label_name:]
process [(identifier {, ...} | all)] [is]
	-- Declarations (including variables)
	{...}
begin
	-- Sequential statements
	{...}
end process [label_name];
\end{lstlisting}

A process gets activated on any event that occurs on the signals listed in its \emph{sensitivity} list. If there is no sensitivity list, one has to use at least one \lstinline!wait! statement inside the body of the process. The usage of a sensitivity list and \lstinline!wait! statements is mutually exclusive. Specifying \lstinline!all! in the sensitivity list of a process in VHDL-2008 is equivalent to explicitly listing all signals the process treats as inputs.

\emph{Concurrent signal assignments} are signal assignment statements that reside inside an architecture body. Accordingly, conditional signal assignments inside an architecture body are called \emph{concurrent conditional signal assignments} and selected signal assignments \emph{concurrent selected signal assignments}. This enables \emph{functional descriptions} of digital systems.

Example of a concurrent selected signal assignment:
\begin{lstlisting}
entity mux4 is
	port (a0, a1, a2, a3 : in bit;
	      sel0, sel1     : in bit;
	      y              : out bit);
	constant T_PD : time := 4.5 ns;
end entity mux4;

architecture behav of mux4 is
begin
	with bit_vector'(sel0, sel1) select
		y <= a0 when "00",
		     a1 when "01",
		     a2 when "10",
		     a3 when "11";
end architecture behav;
\end{lstlisting}

Similarly, assertions inside the body of an architecture or entity are referred to as \emph{concurrent assertion statements}.

A \emph{structural model} of a system can be constructed using \emph{component instantiation statements} inside an architecture body:
\begin{lstlisting}
label_name:
entity identifier [(architecture_identifier)]
	[generic map (
		[identifier =>] expression
		                | open
		{, ...}
	)]

	[port map (
		[identifier =>] signal_identifier
		                | expression
		                | open
		{, ...}
	)]
;
\end{lstlisting}

Specifying \lstinline!open! in the port association list leaves the corresponding port unconnected.
%
%
%
%
\subsection{Procedures and Functions}
Syntax of a \emph{procedure} declaration:
\begin{lstlisting}
procedure identifier
[generic (
	-- C.f. entity declaration
)]
[[parameter] (
	-- Formal parameters
	[constant | variable | signal]
	-- Out parameters may only be read in VHDL-2008
	identifier {, ...} : [in|out|inout] typename
		[:= default_value]
	{; ...}
)] is
	-- Declarations (local to the procedure)
	{...}
begin
	-- Sequential statements
	{...}
end [procedure] [identifier];
\end{lstlisting}

The statement \lstinline![label_name:] return;! returns immediately from a procedure.

Syntax of \emph{procedure calls}:
\begin{lstlisting}
[label_name:]
procedure_identifier[(
	[parameter_identifier =>]
		expression
		| signal_identifier
		| variable_identifier
		| open
	{, ...}
)];
\end{lstlisting}

Example of a procedure:
\begin{lstlisting}
entity sum is
end entity sum;

architecture behav of sum is
begin
	process is
		type real_array is array (natural range <>) of real;
		constant deviations : real_array := (1.0, 2.0, 3.0);
		variable sum_of_squares : real;

		procedure calc_sos (dev : in real_array; sos : out real) is
			variable tmp : real := 0.0;
		begin
			for i in dev'range loop
				tmp := tmp + dev(i)**2;
			end loop;

			sos := tmp;
		end procedure calc_sos;
	begin
		calc_sos(deviations, sum_of_squares);
		report "Sum of squares: " & to_string(sum_of_squares);
		wait;
	end process;
end architecture behav;
\end{lstlisting}

Syntax of \emph{function} declarations:
\begin{lstlisting}
-- Default: pure
[pure | impure]
function identifier
[generic (
	-- C.f. entity declarations
)]
[[parameter] (
	-- Default: constant
	[constant | signal]
	-- Default: in
	identifier {, ...} : [in] typename
		[:= default_value]
	{; ...}
)] return type_name is
	-- Declarations (local to the function)
	{...}
begin
	-- Sequential statements
	{...}
end [function] [identifier];
\end{lstlisting}

\emph{Pure} functions do not read any variables or signals other than those specified in its parameter list; the impure keyword indicates that a function may have side effects. The statement \lstinline![label_name:] return value;! exits the function immediately returning \lstinline!value!. Functions may not include any \lstinline!wait! statements.

Syntax of \emph{function calls}:
\begin{lstlisting}
function_identifier [(
	[parameter_identifier =>]
		expression
		| signal_identifier
		| open
	{, ...}
)];
\end{lstlisting}

The predefined function \lstinline!now! returns the current simulation time as a \lstinline!delay_length!.

Example of a function:
\begin{lstlisting}
entity check_edge is
end entity check_edge;

architecture behav of check_edge is
	function valid_edge(signal x : in std_ulogic) return boolean is
	begin
		if (x'last_value = '0' or x'last_value = 'L')
		   and (x = '1' or x = 'H') then
			return true;
		elsif (x'last_value = '1' or x'last_value = 'H')
		      and (x = '0' or x = 'L') then
			return true;
		else
			return false;
		end if;
	end function valid_edge;

	signal x : std_ulogic := 'U';
begin
	process is
	begin
		x <= '1';
		wait until x = '1';
		report "U -> 1: " & to_string(valid_edge(x));
		x <= '0';
		wait until x = '0';
		report "1 -> 0: " & to_string(valid_edge(x));
		x <= 'L';
		wait until x = 'L';
		report "0 -> L: " & to_string(valid_edge(x));
		x <= 'H';
		wait until x = 'H';
		report "L -> H: " & to_string(valid_edge(x));
		wait;
	end process;
end architecture behav;
\end{lstlisting}

Example of \emph{operator overloading}:
\begin{lstlisting}
entity overloading is
end entity overloading;

architecture behav of overloading is
	function "and" (left, right : integer) return boolean is
		variable l, r : boolean;
	begin
		l := false when left = 0 else true when left /= 0;
		r := false when right = 0 else true when right /= 0;
		return l and r;
	end function "and";

	function "or" (left, right : integer) return boolean is
		variable l, r : boolean;
	begin
		l := false when left = 0 else true when left /= 0;
		r := false when right = 0 else true when right /= 0;
		return l or r;
	end function "or";
begin
	process is
	begin
		report "0 and 1 = " & to_string(0 and 1);
		report "1 and 1 = " & to_string(1 and 1);
		report "0 or  0 = " & to_string(0 or  0);
		report "0 or  1 = " & to_string(0 or  1);
		wait;
	end process;
end architecture behav;
\end{lstlisting}
%
%
%
%
\subsection{FSM}
Possible state encoding formats:

\begin{tabular}{lcccc}
\toprule
\textbf{Decimal} & \textbf{Binary} & \textbf{Gray} & \textbf{Johnson} & \textbf{One-Hot}\\
\midrule
	0  & 0000 & 0000 & 00000000 & 0000000000000001\\
	1  & 0001 & 0001 & 00000001 & 0000000000000010\\
	2  & 0010 & 0011 & 00000011 & 0000000000000100\\
	3  & 0011 & 0010 & 00000111 & 0000000000001000\\
	4  & 0100 & 0110 & 00001111 & 0000000000010000\\
	5  & 0101 & 0111 & 00011111 & 0000000000100000\\
	6  & 0110 & 0101 & 00111111 & 0000000001000000\\
	7  & 0111 & 0100 & 01111111 & 0000000010000000\\
	8  & 1000 & 1100 & 11111111 & 0000000100000000\\
	9  & 1001 & 1101 & 11111110 & 0000001000000000\\
	10 & 1010 & 1111 & 11111100 & 0000010000000000\\
	11 & 1011 & 1110 & 11111000 & 0000100000000000\\
	12 & 1100 & 1010 & 11110000 & 0001000000000000\\
	13 & 1101 & 1011 & 11100000 & 0010000000000000\\
	14 & 1110 & 1001 & 11000000 & 0100000000000000\\
	15 & 1111 & 1000 & 10000000 & 1000000000000000\\
\bottomrule
\end{tabular}

FSM with separate register, next state and output logic:
\begin{lstlisting}
entity fsm1 is
	port (clk, rst : in std_logic;
	      x        : in std_logic;
	      y        : out std_logic);
end entity;

architecture behav of fsm1 is
	type fsm_state is (S0, S1, S2);
	signal state, next_state : fsm_state;
begin
	-- Next state logic
	process (state, x)
	begin
		case state is
			when S0 =>
				next_state <= S1;
			when S1 =>
				if x = '1' then
					next_state <= S2;
				else
					next_state <= S1;
				end if;
			when S2 =>
				next_state <= S0;
			when others =>
				next_state <= S0;
		end case;
	end process;

	-- Register
	process (clk, rst)
	begin
		if rst = '1' then
			state <= S0;
		elsif rising_edge(clk) then
			state <= next_state;
		end if;
	end process;

	-- Moore output (concurrent selected signal assignment)
	with state select y <=
		'1' when S0,
		'1' when S1,
		'0' when S2,
		'0' when others;
end architecture;
\end{lstlisting}

FSM with combined register and next state logic, separate output logic:
\begin{lstlisting}
entity fsm2 is
	port (clk, rst : in std_logic;
	      x        : in std_logic;
	      y        : out std_logic);
end entity;

architecture behav of fsm2 is
	type fsm_state is (S0, S1, S2);
	signal state : fsm_state;
begin
	-- Register, next state logic
	process (clk, rst)
	begin
		if rst = '1' then
			state <= S0;
		elsif rising_edge(clk) then
			case state is
				when S0 =>
					state <= S1;
				when S1 =>
					if x = '1' then
						state <= S2;
					else
						state <= S1;
					end if;
				when S2 =>
					state <= S0;
				when others =>
					null;
			end case;
		end if;
	end process;

	-- Moore output (separate process)
	process (state)
	begin
		case state is
			when S0 =>
				y <= '1';
			when S1 =>
				y <= '1';
			when S2 =>
				y <= '0';
			when others =>
				y <= '0';
		end case;
	end process;
end architecture;
\end{lstlisting}

FSM with combined next state and output logic, separate register:
\begin{lstlisting}
entity fsm3 is
	port (clk, rst : in std_logic;
	      x        : in std_logic;
	      y        : out std_logic);
end entity;

architecture behav of fsm3 is
	type fsm_state is (S0, S1, S2);
	signal state, next_state : fsm_state;
begin
	-- Next state and output logic
	process (state, x)
	begin
		case state is
			when S0 =>
				-- Next state
				state <= S1;
				-- Moore output
				y <= '1';
			when S1 =>
				-- Next state
				if x = '1' then
					state <= S2;
				else
					state <= S1;
				end if;
				-- Moore output
				y <= '1';
			when S2 =>
				-- Next state
				state <= S0;
				-- Moore output
				y <= '0';
			when others =>
				state <= S0;
				y <= '0';
		end case;
	end process;

	-- Register
	process (clk, rst)
	begin
		if rst = '1' then
			state <= S0;
		elsif rising_edge(clk) then
			state <= next_state;
		end if;
	end process;
end architecture;
\end{lstlisting}

FSM with combined register, next state and output logic:
\begin{lstlisting}
entity fsm4 is
	port (clk, rst : in std_logic;
	      x        : in std_logic;
	      y        : out std_logic);
end entity;

architecture behav of fsm4 is
begin
	process (clk, rst)
		type fsm_state is (S0, S1, S2);
		variable state : fsm_state;
	begin
		-- Register, next state logic
		if rst = '1' then
			state := S0;
		elsif rising_edge(clk) then
			case state is
				when S0 =>
					state := S1;
				when S1 =>
					if x = '1' then
						state := S2;
					else
						state := S1;
					end if;
				when S2 =>
					state := S0;
				when others =>
					state := S0;
			end case;
		end if;

		-- Moore output
		case state is
			when S0 =>
				y <= '1';
			when S1 =>
				y <= '1';
			when S2 =>
				y <= '0';
			when others =>
				y <= '0';
		end case;
	end process;
end architecture;
\end{lstlisting}
%
%
%
\subsection{Miscellaneous}
Syntax of \emph{package} declarations:
\begin{lstlisting}
package identifier is
	-- Declarations
	{...}
end [package] [identifier];
\end{lstlisting}

Syntax of \emph{package bodies}:
\begin{lstlisting}
package body identifier is
	-- Declarations
	{...}
end [package body] [identifier];
\end{lstlisting}

Implicit context clause in each design unit:
\begin{lstlisting}
library std, work;
use std.standard.all;
\end{lstlisting}

Declaring aliases for data objects:\\ \lstinline!alias new_identifier is old_identifier;!
%
%
%
%
%
\section{ED}
\subsection{Vector Analysis}
Identities involving vector products ($\vec a, \vec b, \vec c\in\mathbb{R}^3$, the wedge operator $\wedge$ denotes the vector product):
\begin{gather*}
	\vec a\cdot(\vec b\wedge\vec c) = \vec b\cdot(\vec c\wedge\vec a) = \vec c\cdot(\vec a\wedge\vec b)\\
	\vec a\cdot(\vec c\wedge\vec b) = \vec b\cdot(\vec a\wedge\vec c) = \vec c\cdot(\vec b\wedge\vec a)
\end{gather*}
\begin{gather*}
	\vec a\wedge(\vec b\wedge\vec c) = \vec b(\vec a\cdot\vec c) - \vec c(\vec a\cdot\vec b)\\
	(\vec a\wedge\vec b)\wedge\vec c = \vec b(\vec a\cdot\vec c) - \vec a(\vec b\cdot\vec c)
\end{gather*}

Definition of the \emph{nabla} (\emph{del}) operator:
\[
	\vec\nabla = \begin{bmatrix}\nicefrac{\partial}{\partial x}\\
	                            \nicefrac{\partial}{\partial y}\\
	                            \nicefrac{\partial}{\partial z}
	             \end{bmatrix}
\]

\emph{Gradient} of a scalar field $f$:
\[
	\grad(f) = \vec\nabla f = \begin{bmatrix}\nicefrac{\partial f}{\partial x}\\
	                                         \nicefrac{\partial f}{\partial y}\\
	                                         \nicefrac{\partial f}{\partial z}
	                          \end{bmatrix}
\]

\emph{Divergence} of a vector field $\vec F$:
\[
	\Div(\vec F) = \vec\nabla\cdot\vec F = \frac{\partial F_x}{\partial x}
	                                       + \frac{\partial F_y}{\partial y}
	                                       + \frac{\partial F_z}{\partial z}
\]

\emph{Curl} of a vector field $\vec F$:
\[
	\curl(\vec F) = \vec\nabla\wedge\vec F = \begin{vmatrix}\hat e_x & \hat e_y & \hat e_z\\
	                                                        \nicefrac{\partial}{\partial x} & \nicefrac{\partial}{\partial y} & \nicefrac{\partial}{\partial z}\\
	                                                        F_x & F_y & F_z
	                                         \end{vmatrix}
\]

Product rules for gradients ($f,g$: scalar fields; $\vec F,\vec G$: vector fields):
\begin{gather*}
	\vec\nabla(fg) = f(\vec\nabla g) + g(\vec\nabla f)\\
	\vec\nabla(\vec F\cdot\vec G) = \vec F\wedge(\vec\nabla\wedge\vec G) + \vec G\wedge(\vec\nabla\wedge\vec F)
	                               +(\vec F\cdot\vec\nabla)\vec G + (\vec G\cdot\vec\nabla)\vec F
\end{gather*}

Product rules for divergences ($f$: scalar field; $\vec F,\vec G$: vector fields):
\begin{gather*}
	\vec\nabla\cdot(f\vec G) = f(\vec\nabla\cdot\vec G) + \vec G\cdot(\vec\nabla f)\\
	\vec\nabla\cdot(\vec F\wedge\vec G) = \vec G\cdot(\vec\nabla\wedge\vec F) - \vec F\cdot(\vec\nabla\wedge\vec G)
\end{gather*}

Product rules for curls ($f$: scalar field; $\vec F,\vec G$: vector fields):
\begin{gather*}
	\vec\nabla\wedge(f\vec G) = f(\vec\nabla\wedge\vec G) - \vec G\wedge(\vec\nabla f)\\
	\vec\nabla\wedge(\vec F\wedge\vec G) = (\vec G\cdot\vec\nabla)\vec F - (\vec F\cdot\vec\nabla)\vec G+\vec F(\vec\nabla\cdot\vec G) - \vec G(\vec\nabla\cdot\vec F)
\end{gather*}

Definition of the \emph{Laplacian} ($f$: scalar field):
\[
	\vec\nabla\cdot(\vec\nabla f) = \vec\nabla^2 f = \Delta f = \frac{\partial^2 f}{\partial x^2} + \frac{\partial^2 f}{\partial y^2} + \frac{\partial^2 f}{\partial z^2}
\]

The curl of a gradient and the divergence of a curl always vanish:
\begin{gather*}
	\vec\nabla\wedge(\vec\nabla f)\equiv\vec 0\\
	\vec\nabla\cdot(\vec\nabla\wedge\vec F)\equiv 0
\end{gather*}

Curl of a curl:
\[
	\vec\nabla\wedge(\vec\nabla\wedge\vec F) = \vec\nabla(\vec\nabla\cdot\vec F) - \vec\nabla^2\vec F
\]

Fundamental theorem for gradients ($f$: scalar field):
\[
	\int_{\vec a}^{\vec b} (\vec\nabla f)\cdot\mathrm{d}\vec r = f(\vec b) - f(\vec a)
\]

\emph{Gauss's theorem} ($\vec F$: vector field):
\[
	\int_V(\vec\nabla\cdot\vec F)\,\mathrm{d}V = \oint_{\partial V}\vec F\cdot\mathrm{d}\vec a
\]

\emph{Stokes' theorem} ($\vec F$: vector field):
\[
	\int_\mathcal{S}(\vec\nabla\wedge\vec F)\cdot\mathrm{d}\vec a = \oint_{\partial\mathcal{S}}\vec F\cdot\mathrm{d}\vec r
\]

\emph{Spherical} coordinates:
\[
	\begin{bmatrix}x\\ y\\ z\end{bmatrix} =
	r\cdot\begin{bmatrix}\sin(\vartheta)\cdot\cos(\varphi)\\ \sin(\vartheta)\cdot\sin(\varphi)\\ \cos(\vartheta)\end{bmatrix}
\]

Unit vectors:
\begin{align*}
	\hat e_r &= \sin(\vartheta)\cdot\cos(\varphi)\cdot\hat e_x + \sin(\vartheta)\cdot\sin(\varphi)\cdot\hat e_y + \cos(\vartheta)\cdot\hat e_z\\
	\hat e_\vartheta &= \cos(\vartheta)\cdot\cos(\varphi)\cdot\hat e_x + \cos(\vartheta)\cdot\sin(\varphi)\cdot\hat e_y - \sin(\vartheta)\cdot\hat e_z\\
	\hat e_\varphi &= -\sin(\varphi)\cdot\hat e_x + \cos(\varphi)\cdot\hat e_y
\end{align*}

Line and volume elements:
\begin{align*}
	\mathrm{d}\vec r &= \mathrm{d}r\cdot\hat e_r + r\,\mathrm{d}\vartheta\cdot\hat e_\vartheta + r\sin(\vartheta)\,\mathrm{d}\varphi\cdot\hat e_\varphi\\
	\mathrm{d}V &= r^2\sin(\vartheta)\,\mathrm{d}r\,\mathrm{d}\vartheta\,\mathrm{d}\varphi
\end{align*}

Gradient:
\[
	\vec\nabla f = \frac{\partial f}{\partial r}\cdot\hat e_r + \frac{1}{r}\,\frac{\partial f}{\partial \vartheta}\cdot\hat e_\vartheta
	               + \frac{1}{r\sin(\vartheta)}\,\frac{\partial f}{\partial\varphi}\cdot\hat e_\varphi
\]

Divergence:
\[
	\vec\nabla\cdot\vec F = \frac{1}{r^2}\,\frac{\partial}{\partial r}(r^2\cdot F_r)
	                        + \frac{1}{r\sin(\vartheta)}\,\frac{\partial}{\partial\vartheta}\left(\sin(\vartheta)\cdot F_\vartheta\right)
	                        + \frac{1}{r\sin(\vartheta)}\,\frac{\partial}{\partial\varphi} F_\varphi
\]

Curl:
\begin{equation*}
\begin{split}
	\vec\nabla\wedge\vec F &= \frac{1}{r\sin(\vartheta)}\cdot\left[\frac{\partial}{\partial\vartheta}\left(\sin(\vartheta)\cdot F_\varphi\right)
	                    - \frac{\partial}{\partial\varphi}F_\vartheta\right]\cdot\hat e_r\\
	                 &\quad +\frac{1}{r}\cdot\left[\frac{1}{\sin(\vartheta)}\,\frac{\partial}{\partial\varphi}F_r
	                    - \frac{\partial}{\partial r}(r\cdot F_\varphi)\right]\cdot\hat e_\vartheta\\
	                 &\quad +\frac{1}{r}\cdot\left[\frac{\partial}{\partial r}(r\cdot F_\vartheta)-\frac{\partial}{\partial\vartheta}F_r\right]\cdot\hat e_\varphi
\end{split}
\end{equation*}

Laplacian:
\begin{equation*}
	\Delta f = \frac{1}{r^2}\,\frac{\partial}{\partial r}\left(r^2\cdot\frac{\partial f}{\partial r}\right)
	           + \frac{1}{r^2\sin(\vartheta)}\,\frac{\partial}{\partial\vartheta}\left(\sin(\vartheta)\cdot\frac{\partial f}{\partial\vartheta}\right)
	           + \frac{1}{r^2\sin^2(\vartheta)}\cdot\frac{\partial^2 f}{\partial\varphi^2}
\end{equation*}

\emph{Cylindrical} coordinates:
\[
	\begin{bmatrix}x\\ y\\ z\end{bmatrix} =
	\begin{bmatrix}\varrho\cdot\cos(\varphi)\\ \varrho\cdot\sin(\varphi)\\ z\end{bmatrix}
\]

Unit vectors:
\begin{align*}
	\hat e_\varrho &= \cos(\varphi)\cdot\hat e_x + \sin(\varphi)\cdot\hat e_y\\
	\hat e_\varphi &= -\sin(\varphi)\cdot\hat e_x + \cos(\varphi)\cdot\hat e_y\\
	\hat e_z &= \hat e_z
\end{align*}

Line and volume elements:
\begin{align*}
	\mathrm{d}\vec r &= \mathrm{d}\varrho\cdot\hat e_\varrho + \varrho\,\mathrm{d}\varphi\cdot\hat e_\varphi + \mathrm{d}z\cdot\hat e_z\\
	\mathrm{d}V &= \varrho\,\mathrm{d}\varrho\,\mathrm{d}\varphi\,\mathrm{d}z
\end{align*}

Gradient:
\[
	\vec\nabla f = \frac{\partial f}{\partial\varrho}\cdot\hat e_\varrho + \frac{1}{\varrho}\,\frac{\partial f}{\partial\varphi}\cdot\hat e_\varphi
	               + \frac{\partial f}{\partial z}\cdot\hat e_z
\]

Divergence:
\[
	\vec\nabla\cdot\vec F = \frac{1}{\varrho}\,\frac{\partial}{\partial\varrho}(\varrho\cdot F_\varrho)
	                        + \frac{1}{\varrho}\,\frac{\partial}{\partial\varphi}\cdot F_\varphi + \frac{\partial}{\partial z}\cdot F_z
\]

Curl:
\begin{equation*}
\begin{split}
	\vec\nabla\wedge\vec F &= \left(\frac{1}{\varrho}\,\frac{\partial}{\partial\varphi}\cdot F_z-\frac{\partial}{\partial z}\cdot F_\varphi\right)\cdot\hat e_\varrho\\
	                       &\quad+ \left(\frac{\partial}{\partial z}F_\varrho-\frac{\partial}{\partial\varrho}\cdot F_z\right)\cdot\hat e_\varphi\\
	                       &\quad+ \frac{1}{\varrho}\cdot\left[\frac{\partial}{\partial\varrho}(\varrho\cdot F_\varphi)-\frac{\partial}{\partial\varphi}\cdot
	                               F_\varrho\right]\cdot\hat e_z
\end{split}
\end{equation*}

Laplacian:
\[
	\Delta f = \frac{1}{\varrho}\,\frac{\partial}{\partial\varrho}\left(\varrho\cdot\frac{\partial f}{\partial\varrho}\right)
	           + \frac{1}{\varrho^2}\cdot\frac{\partial^2 f}{\partial\varphi^2} + \frac{\partial^2 f}{\partial z^2}
\]

\emph{Delta distribution} (differentiation with respect to $\vec r$, $\vec r'\equiv\text{const.}$):
\begin{gather*}
	\vec\nabla\cdot\left(\frac{\vec r-\vec r'}{|\vec r-\vec r'|^3}\right) = 4\pi\cdot\delta^{(3)}(\vec r-\vec r')\\
	\Delta\left(\frac{1}{|\vec r-\vec r'|}\right) = -4\pi\cdot\delta^{(3)}(\vec r-\vec r')
\end{gather*}

Properties of \emph{curl-less} fields:
\begin{enumerate}
	\item $\vec\nabla\wedge\vec F\equiv\vec 0$
	\item $\int_\gamma\vec F\cdot\mathrm{d}\vec r$ is independent of the path $\gamma$
	\item $\oint_\gamma\vec F\cdot\mathrm{d}\vec r\equiv 0\quad\forall\gamma$
	\item $\vec F=-\vec\nabla \phi\quad$ ($\phi$: scalar potential)
\end{enumerate}

Properties of \emph{divergence-less} field:
\begin{enumerate}
	\item $\vec\nabla\cdot\vec F\equiv 0$
	\item $\int_\mathcal{S}\vec F\cdot\mathrm{d}\vec a$ is independent of the surface $\mathcal{S}$
	\item $\oint_\mathcal{S}\vec F\cdot\mathrm{d}\vec a\equiv 0\quad\forall \mathcal{S}$
	\item $\vec F=\vec\nabla\wedge\vec A\quad$ ($\vec A$: vector potential)
\end{enumerate}
%
%
%
%
\subsection{Electrostatics}
\emph{Coulomb's law} (force on a test charge $Q$ caused by a point charge $q$ at location $\vec r'$,\\ $\varepsilon_0$: permittivity of free space):
\[
	\vec F(\vec r) = \frac{qQ}{4\pi\varepsilon_0}\cdot\frac{\vec r-\vec r'}{|\vec r-\vec r'|^3}
\]

Force on a test charge exerted by several point charges $q_1,\dots,q_n$:
\[
	\vec F = Q\cdot\vec E
\]

\emph{Electric field}:
\[
	\vec E(\vec r) = \frac{1}{4\pi\varepsilon_0}\cdot\sum_{j=1}^n q_j\cdot\frac{\vec r-\vec r_j}{|\vec r-\vec r_j|^3}
\]

Electric field generated by the charge distribution $\varrho(\vec r)$:
\[
	\vec E(\vec r) = \frac{1}{4\pi\varepsilon_0}\cdot\int\varrho(\vec r')\cdot\frac{\vec r-\vec r'}{|\vec r-\vec r'|^3}\,\mathrm{d}^3 r'
\]

\emph{Electric flux} through a surface $\mathcal{S}$:
\[
	\Phi = \int_\mathcal{S}\vec E\cdot\mathrm{d}\vec a
\]

\emph{Gauss's law} (differential form):
\begin{equation*}
\begin{split}
	\vec\nabla\cdot\vec E &= \frac{1}{4\pi\varepsilon_0}\cdot\int\mathrm{d}^3 r'\,\varrho(\vec r')\,\vec\nabla\cdot\left(\frac{\vec r-\vec r'}{|\vec r-\vec r'|^3}\right)
	                         = \frac{1}{4\pi\varepsilon_0}\cdot\int\mathrm{d}^3 r'\,\varrho(\vec r')\,4\pi\cdot\delta^{(3)}(\vec r-\vec r')\\
	                      &= \frac{1}{\varepsilon_0}\cdot\varrho(\vec r)
\end{split}
\end{equation*}

Gauss's law (integral form):
\begin{equation*}
\begin{split}
	\oint_{\partial\mathcal{V}}\vec E\cdot\mathrm{d}\vec a &= \int_\mathcal{V}\vec\nabla\cdot\vec E\,\mathrm{d}V = \frac{1}{\varepsilon_0}\cdot\int_\mathcal{V}\varrho(\vec r)\,\mathrm{d}V\\
	                                              &= \frac{1}{\varepsilon_0}\cdot Q_\mathrm{total}
\end{split}
\end{equation*}

Closed paths, curl of an electric field:
\begin{gather*}
		\oint_\gamma\vec E\cdot\mathrm{d}\vec r\equiv 0\quad\forall\gamma\\
		\vec\nabla\wedge\vec E\equiv\vec 0
\end{gather*}

Definition of the electric potential ($\vec r_\mathrm{ref}$: reference point):
\begin{equation*}
	\phi(\vec r)=-\int_{\vec r_\mathrm{ref}}^{\vec r}\vec E(\vec r')\cdot\mathrm{d}\vec r'\,\curvearrowright\,\vec E(\vec r)=-\vec\nabla\phi(\vec r)
\end{equation*}

Poisson's equation:
\[
	\vec\nabla^2\phi=\Delta\phi=-\frac{\varrho}{\varepsilon_0}
\]

Laplace's equation ($\varrho\equiv 0$):
\[
	\vec\nabla^2\phi=\Delta\phi=0
\]

Potential of a charge distribution ($\vec r_\mathrm{ref}\to\infty$):
\[
	\phi(\vec r)=\frac{1}{4\pi\varepsilon_0}\cdot\int\frac{\varrho(\vec r')}{|\vec r-\vec r'|}\,\mathrm{d}^3 r'
\]

Energy stored in an electric field:
\[
	W=\frac{\varepsilon_0}{2}\cdot\int_{\mathbb{R}^3}\vec E^2\,\mathrm{d}V
\]

Properties of an ideal conductor:
\begin{enumerate}
	\item $\vec E\equiv\vec 0$ on the inside
	\item $\varrho\equiv 0$ on the inside
	\item Any net charge resides on the surface.
	\item $\phi(\vec r)\equiv\text{const.}$
	\item $\vec E\perp \text{surface}$
\end{enumerate}

Boundary conditions at the proximity of a surface with surface charge density $\sigma$ ($\hat n$: unit vector perpendicular to the surface):
\[
	\vec E_\mathrm{above}-\vec E_\mathrm{below}=\frac{\sigma}{\varepsilon_0}\cdot\hat n
\]

Multipole expansion in cartesian coordinates ($P_n$: Legendre polynomials, $\alpha:=\angle\left(\vec r,\vec r'\right)$):
\[
	\phi(\vec r)=\frac{1}{4\pi\varepsilon_0}\cdot\sum_{n=0}^\infty\frac{1}{|\vec r|^{n+1}}\,\int|\vec r'|^n\,P_n\left(\cos(\alpha)\right)\,\varrho(\vec r')\,\mathrm{d}^3 r'
\]

Dipole moments:
\begin{gather*}
	\vec p=\int\vec r'\,\varrho(\vec r')\,\mathrm{d}^3 r'\\
	\phi^{(2)}(\vec r)=\frac{1}{4\pi\varepsilon_0}\cdot\frac{\vec p\cdot\hat e_r}{|\vec r|^2}
\end{gather*}
%
%
%
%
\subsection{Electric fields in matter}
Polarization:
\[
	\vec P=\frac{\text{dipole moment}}{\text{unit volume}}
\]

Electric displacement:
\begin{gather*}
	\vec D=\varepsilon_0\,\vec E+\vec P\\
	\vec\nabla\wedge\vec D=\varepsilon_0\,(\vec\nabla\wedge\vec E)+\vec\nabla\wedge\vec P=
		\vec\nabla\wedge\vec P\neq\vec 0
\end{gather*}

Gauss's law ($\varrho_\mathrm{f}$: free charge density):
\begin{gather*}
	\vec\nabla\cdot\vec D = \varrho_\mathrm{f}\\
	\oint_{\partial\mathcal{V}}\vec D\cdot\mathrm{d}\vec a = Q_\mathrm{total}
\end{gather*}

Boundary conditions ($\sigma_\mathrm{f}$: free surface charge density):
\begin{gather*}
	D_\text{above}^\perp -D_\text{below}^\perp = \sigma_\mathrm{f}\\
	D_\text{above}^\parallel - D_\text{below}^\parallel = P_\text{above}^\parallel - P_\text{below}^\parallel
\end{gather*}

Polarization in linear dielectrics ($\chi_\mathrm{e}$: electric susceptibility):
\[
	\vec P=\varepsilon_0\chi_\mathrm{e}\,\vec E
\]

Permittivity:
\[
	\varepsilon=\varepsilon_0\,(1+\chi_\mathrm{e})
\]

Electric displacement:
\[
	\vec D=\varepsilon\,\vec E
\]

Dielectric constant:
\[
	\varepsilon_\mathrm{r} = 1+\chi_\mathrm{e}=\frac{\varepsilon}{\varepsilon_0}
\]
%
%
%
%
\subsection{Magnetostatics}
Lorentz force law:
\[
	\vec F=q\cdot(\vec E+\vec v\wedge\vec B)
\]

Force on a wire with impressed current $I$ in a magnetic field $\vec B$:
\[
	\vec F_\mathrm{mag}=\int I\,(\mathrm{d}\vec l\wedge\vec B)
\]

Surface current density ($\sigma$: surface charge density):
\[
	\vec K=\frac{\mathrm{d}\vec I}{\mathrm{d}l_\perp}=\sigma\vec v
\]

Force on a surface current in a magnetic field $\vec B$:
\[
	\vec F_\mathrm{mag}=\int(\vec K\wedge\vec B)\,\mathrm{d}a
\]

Volume current density ($\varrho$: volume charge density):
\[
	\vec J=\frac{\mathrm{d}\vec I}{\mathrm{d}a_\perp}=\varrho\vec v
\]

Force on a volume current in a magnetic field $\vec B$:
\[
	\vec F_\mathrm{mag}=\int(\vec I\wedge\vec B)\,\mathrm{d}V
\]

Continuity equation (local charge conservation):
\[
	\vec\nabla\cdot\vec J=-\frac{\partial\varrho}{\partial t}
\]

Biot-Savart law (magnetic field of a steady line current):
\[
	\vec B(\vec r)=\frac{\mu_0}{4\pi}\cdot\int\vec I\wedge\frac{\vec r-\vec r'}{|\vec r-\vec r'|^3}\,\mathrm{d}l'\\
		=\frac{\mu_0}{4\pi}\,I\cdot\int\mathrm{d}\vec l'\wedge\frac{\vec r-\vec r'}{|\vec r-\vec r'|^3}
\]

Ampere's law (differential form):
\[
	\vec\nabla\wedge\vec B=\mu_0\cdot\vec J
\]

Ampere's law (integral form, $I_\mathrm{total}$: total enclosed current):
\[
	\oint\vec B\cdot\mathrm{d}\vec l=\mu_0\cdot I_\mathrm{total}
\]

Definition of the vector potential:
\begin{gather*}
	\vec B=\vec\nabla\wedge\vec A\\
	\vec\nabla\cdot\vec A\equiv 0
\end{gather*}

Ampere's law:
\[
	\vec\nabla^2\vec A=-\mu_0\cdot\vec J
\]

General solution ($\vec J(|\vec r|\to\infty)\to\vec 0$):
\[
	\vec A(\vec r)=\frac{\mu_0}{4\pi}\cdot\int\frac{\vec J(\vec r')}{|\vec r-\vec r'|}\,\mathrm{d}^3 r'
\]

Boundary conditions at a surface current ($\hat n$: unit vector perpendicular to the surface):
\[
	\vec B_\mathrm{above}-\vec B_\mathrm{below}=\mu_0\cdot\vec K\wedge\hat n
\]

Magnetic dipole moment:
\[
	\vec m=I\,\int\mathrm{d}\vec a=I\,\vec a
\]

Vector potential of a magnetic dipole:
\[
	\vec A^{(2)}(\vec r)=\frac{\mu_0}{4\pi}\cdot\frac{\vec m\wedge\hat e_r}{|\vec r|^2}
\]
%
%
%
%
\subsection{Magnetic Fields in Matter}
Magnetization:
\[
	\vec M=\frac{\text{magnetic dipole moment}}{\text{unit volume}}
\]

Definition of the magnetic field:
\[
	\vec H=\frac{1}{\mu_0}\cdot\vec B-\vec M
\]

Ampere's law ($\vec J_\mathrm{f}$: free current, $I_\mathrm{total}$: total free current):
\begin{gather*}
	\vec\nabla\wedge\vec H=\vec J_\mathrm{f}\\
	\oint\vec H\cdot\mathrm{d}\vec l=I_\mathrm{total}
\end{gather*}

Divergence of the magnetic field:
\[
	\vec\nabla\cdot\vec H=-\vec\nabla\cdot\vec M
\]

Boundary conditions at a surface current ($\hat n$: unit vector perpendicular to the surface, $\vec K_\mathrm{f}$: free surface current):
\[
	\vec H_\mathrm{above}^\parallel-\vec H_\mathrm{below}^\parallel=\vec K_\mathrm{f}\wedge\hat n
\]

Magnetization in linear matter ($\chi_\mathrm{m}$: magnetic susceptibility):
\[
	\vec M=\chi_\mathrm{m}\cdot\vec H
\]

Magnetic flux density in linear matter ($\mu$: permeability, $\mu_\mathrm{r}$: relative permeability,\\ $\mu_0$: permeability of free space):
\[
	\vec B=\mu_0\cdot(\vec H+\vec M)=\mu_0\cdot(1+\chi_\mathrm{m})\cdot\vec H=\mu_0\,\mu_\mathrm{r}\cdot\vec H=\mu\cdot\vec H
\]
%
%
%
%
\subsection{Electrodynamics}
Ohm's law ($\sigma$: conductivity, $\varrho=\nicefrac{1}{\sigma}$: resistivity):
\[
	\vec J=\sigma\,\vec E
\]

Resistance:
\[
	R=\frac{V}{I}
\]

Electromotive force (emf, $\vec f_\mathrm{s}$: source):
\[
	\mathcal{E}=\oint \vec f_\mathrm{s}\cdot\mathrm{d}\vec l
\]

Magnetic flux through a surface $\mathcal{S}$:
\[
	\Phi=\int_\mathcal{S}\vec B\cdot\mathrm{d}\vec a
\]

Flux rule for motional emf:
\[
	\mathcal{E}=-\frac{\mathrm{d}\Phi}{\mathrm{d}t}
\]

Faraday's law (differential form):
\[
	\vec\nabla\wedge\vec E=-\frac{\partial\vec B}{\partial t}
\]

Mutual inductance of two loops ($\Phi_1, I_1$: flux and current through loop one; $\Phi_2, I_2$: flux and current through loop two):
\[
	\Phi_2=M_{21}\,I_1
\]

Neumann formula:
\begin{gather*}
	M_{21}=\frac{\mu_0}{4\pi}\cdot\oint\oint\frac{\mathrm{d}\vec I_1\cdot\mathrm{d}\vec I_2}{|\vec r_1-\vec r_2|}\\
	\curvearrowright\,\,M_{21}=M_{12}
\end{gather*}

Self inductance:
\[
	\Phi=L\,I
\]

Energy of a magnetic field:
\[
	W=\int_{\mathbb{R}^3}\vec B^2\,\mathrm{d}V
\]

Maxwell's equations:
\begin{align*}
	\vec\nabla\cdot\vec E&=\frac{1}{\varepsilon_0}\cdot\varrho\\
	\vec\nabla\cdot\vec B&=0\\
	\vec\nabla\wedge\vec E&=-\frac{\partial\vec B}{\partial t}\\
	\vec\nabla\wedge\vec B&=\mu_0\cdot\vec J+\mu_0\,\varepsilon_0\cdot\frac{\partial\vec E}{\partial t}
\end{align*}

Maxwell's equations in matter:
\begin{align*}
	\vec\nabla\cdot\vec D&=\varrho_\mathrm{free}\\
	\vec\nabla\cdot\vec B&=0\\
	\vec\nabla\wedge\vec E&=-\frac{\partial\vec B}{\partial t}\\
	\vec\nabla\wedge\vec H&=\vec J_\mathrm{free}+\frac{\partial\vec D}{\partial t}
\end{align*}

General boundary conditions at a surface ($\hat n$: unit vector perpendicular to the surface, $\sigma_\mathrm{f}$: free surface charge density, $\vec K_\mathrm{f}$: free surface current density):
\begin{gather*}
	D_1^\perp-D_2^\perp=\sigma_\mathrm{f}\\
	B_1^\perp-B_2^\perp=0\\
	\vec E_1^\parallel-\vec E_2^\parallel=\vec 0\\
	\vec H_1^\parallel-\vec H_2^\parallel=\vec K_\mathrm{f}\wedge\hat n
\end{gather*}

General boundary conditions at a surface in linear matter:
\begin{gather*}
	\varepsilon_1\cdot\vec E_1^\perp-\varepsilon_2\cdot\vec E_2^\perp=\sigma_\mathrm{f}\\
	B_1^\perp-B_2^\perp=0\\
	\vec E_1^\parallel-\vec E_2^\parallel=\vec 0\\
	\frac{1}{\mu_1}\cdot\vec B_1^\parallel-\frac{1}{\mu_2}\cdot\vec B_2^\parallel=\vec K_\mathrm{f}\wedge\hat n
\end{gather*}

Energy density in an electromagnetic field:
\[
	u=\frac{1}{2}\,\left(\varepsilon_0\cdot\vec E^2+\frac{1}{\mu_0}\cdot\vec B^2\right)
\]

Poynting vector (energy flux density):
\[
	\vec S=\frac{1}{\mu_0}\,(\vec E\wedge\vec B)
\]

Poynting's theorem:
\[
	\frac{\mathrm{d}W}{\mathrm{d}t}=-\frac{\mathrm{d}}{\mathrm{d}t}\,\int_\mathcal{V}u\,\mathrm{d}V-\oint_{\partial\mathcal{V}}\vec S\cdot\mathrm{d}\vec a
\]

General wave equation:
\[
	\vec\nabla^2 f=\frac{1}{v_\mathrm{ph}^2}\cdot\frac{\partial^2 f}{\partial t^2}
\]

Maxwell's equations in a vacuum:
\begin{align*}
	\vec\nabla\cdot\vec E&=0\\
	\vec\nabla\cdot\vec B&=0\\
	\vec\nabla\wedge\vec E&=-\frac{\partial\vec B}{\partial t}\\
	\vec\nabla\wedge\vec B&=\mu_0\,\varepsilon_0\cdot\frac{\partial\vec E}{\partial t}
\end{align*}

Derivation of the wave equations for $\vec E$ and $\vec B$:
\begin{gather*}
	\vec\nabla\wedge(\vec\nabla\wedge\vec E)=
		\vec\nabla\underbrace{(\vec\nabla\cdot\vec E)}_{=0}-\vec\nabla^2\vec E=-\frac{\partial}{\partial t}(\vec\nabla\wedge\vec B)=-\mu_0\,\varepsilon_0\cdot
		\frac{\partial^2\vec E}{\partial t^2}\\
	\curvearrowright\,\,\vec\nabla^2\vec E=\mu_0\,\varepsilon_0\cdot\frac{\partial^2\vec E}{\partial t^2}\\
	\vec\nabla\wedge(\vec\nabla\wedge\vec B)=
		\vec\nabla\underbrace{(\vec\nabla\cdot\vec B)}_{=0}-\vec\nabla^2\vec B=\mu_0\varepsilon_0\cdot\frac{\partial}{\partial t}(\vec\nabla\wedge\vec E)
		=-\mu_0\,\varepsilon_0\cdot\frac{\partial^2\vec B}{\partial t^2}\\
	\curvearrowright\,\,\vec\nabla^2\vec B=\mu_0\,\varepsilon_0\cdot\frac{\partial^2\vec B}{\partial t^2}
\end{gather*}

Speed of light in a vacuum:
\[
	c_0=\frac{1}{\sqrt{\mu_0\,\varepsilon_0}}
\]

(Complex) plane waves ($\vec k$: wave vector, $|\vec k|=\nicefrac{2\pi}{\lambda}$: wave number, $\hat n$: polarization vector,\\ $\omega=c\cdot k$):
\begin{align*}
	\underline{\vec E}(\vec r,t)&=\underline{E_0}\cdot\mathrm{e}^{\mathrm{i}\,(\vec k\cdot\vec r-\omega\,t)}\cdot\hat n\\
	\underline{\vec B}(\vec r,t)&=\frac{1}{c}\cdot\underline{E_0}\cdot\mathrm{e}^{\mathrm{i}\,(\vec k\cdot\vec r-\omega\,t)}\cdot(\hat e_k\wedge\hat n)
		=\frac{1}{c}\cdot(\hat e_k\wedge\underline{\vec E})
\end{align*}

Maxwell's equations in linear matter in absence of free charge and current:
\begin{align*}
	\vec\nabla\cdot\vec E&=0\\
	\vec\nabla\cdot\vec B&=0\\
	\vec\nabla\wedge\vec E&=-\frac{\partial\vec B}{\partial t}\\
	\vec\nabla\wedge\vec B&=\mu\,\varepsilon\cdot\frac{\partial\vec E}{\partial t}
\end{align*}

Speed of light in linear matter:
\[
	c=\frac{1}{\sqrt{\mu\,\varepsilon}}=\frac{c_0}{n}
\]

Index of refraction:
\[
	n=\sqrt{\mu_\mathrm{r}\,\varepsilon_\mathrm{r}}
\]

Boundary conditions on the inside of a wave guide:
\begin{gather*}
	\vec E^\parallel=\vec 0\\
	B^\perp=0
\end{gather*}

Plane electromagnetic wave travelling in $z$-direction:
\begin{align*}
	\underline{\vec E}&=\underline{\vec E_0}\cdot\mathrm{e}^{\mathrm{i}\,(k\,z-\omega\,t)}\\
	\underline{\vec B}&=\underline{\vec B_0}\cdot\mathrm{e}^{\mathrm{i}\,(k\,z-\omega\,t)}
\end{align*}

Complex amplitudes:
\begin{equation*}
	\underline{\vec E_0}=\begin{bmatrix}E_{0,x}\\E_{0,y}\\E_{0,z}\end{bmatrix};\quad
	\underline{\vec B_0}=\begin{bmatrix}B_{0,x}\\B_{0,y}\\B_{0,z}\end{bmatrix}
\end{equation*}

Maxwell's equations on the inside:
\begin{align*}
	\vec\nabla\cdot\vec E&=0\\
	\vec\nabla\cdot\vec B&=0\\
	\vec\nabla\wedge\vec E&=-\frac{\partial\vec B}{\partial t}\\
	\vec\nabla\wedge\vec B&=\frac{1}{c^2}\cdot\frac{\partial\vec E}{\partial t}
\end{align*}
%
%
%
%
%
\section{QM}
Schroedinger equation in one dimension:
\[
	\mathrm{i}\hbar\,\frac{\partial\Psi}{\partial t}=-\frac{\hbar^2}{2m}\,\frac{\partial^2\Psi}{\partial x^2}+V\cdot\Psi
\]

Statistical interpretation:
\[
	P(a\le x\le b; t)=\int_a^b |\Psi(x,t)|^2\,\mathrm{d}x
\]

Expectation value ($\varrho$: probability density):
\[
	\langle f(x)\rangle=\int_{-\infty}^\infty f(x)\cdot\varrho(x)\,\mathrm{d}x
\]

Variance:
\begin{gather*}
	\sigma^2\equiv\langle\left(x-\langle x\rangle\right)^2\rangle\\
	\sigma^2=\langle x^2\rangle - \langle x\rangle^2
\end{gather*}

Normalization requirement:
\[
	\int_{-\infty}^\infty |\Psi(x,t)|^2\,\mathrm{d}x = 1
\]

Uncertainty principle:
\[
	\sigma_x\,\sigma_p\ge\frac{\hbar}{2}
\]

Hamiltonian ($p\to\frac{\hbar}{\mathrm{i}}\,\frac{\partial}{\partial x}$):
\[
	H=\frac{p^2}{2m}+V(x)=-\frac{\hbar^2}{2m}\,\frac{\partial^2}{\partial x^2}+V(x)
\]

Time-independent Schroedinger equation:
\[
	H\psi(x)=E\psi(x)
\]

General solution of the separable time-dependent Schroedinger equation:
\[
	\Psi(x,t)=\sum_{n=1}^\infty c_n\cdot\psi_n(x)\cdot\mathrm{e}^{-\mathrm{i} E_n\cdot t/\hbar}
\]

1D infinite square well:
\begin{gather*}
	V(x)=\begin{cases}
		0, & 0\le x\le a\\
		\infty, & \text{otherwise}
	\end{cases}\\
	E_n = \frac{n^2\pi^2\hbar^2}{2ma^2}, \quad n=1,2,\dots\\
	\psi_n = \sqrt{\frac{2}{a}}\cdot\sin\left(\frac{n\pi}{a}\cdot x\right)
\end{gather*}

Hermite polynomials (Rodrigues formula):
\[
	H_n(\xi)=(-1)^n\cdot\mathrm{e}^{\xi^2}\cdot\left(\frac{\mathrm{d}}{\mathrm{d}\xi}\right)^n\,\mathrm{e}^{-\xi^2}
\]

1D harmonic oscillator ($\xi\equiv\sqrt{\frac{m\omega}{\hbar}}\cdot x$, $H_n$: Hermite polynomials):
\begin{gather*}
	V(x) = \frac{1}{2} m\omega^2 x^2\\
	E_n = \left(n+\frac{1}{2}\right)\cdot\hbar\omega\quad n=0,1,2,\dots\\
	\psi_n(x) = \left(\frac{m\omega}{\pi\hbar}\right)^{\nicefrac{1}{4}}\cdot\frac{1}{\sqrt{2^n n!}}\cdot
		H_n(\xi)\cdot\mathrm{e}^{-\xi^2/2}
\end{gather*}

Free particle ($k\equiv\pm\frac{\sqrt{2mE}}{\hbar}$, dispersion relation: $\omega=\frac{\hbar k^2}{2m}$):
\begin{gather*}
	\Psi(x,t)=\frac{1}{\sqrt{2\pi}}\cdot\int_{-\infty}^\infty \phi(k)\cdot
		\mathrm{e}^{\mathrm{i}\left(kx-\frac{\hbar k^2}{2m}\cdot t\right)}\,\mathrm{d}k
		=\frac{1}{\sqrt{2\pi}}\cdot\int_{-\infty}^\infty \phi(k)\cdot
		\mathrm{e}^{\mathrm{i}(kx-\omega t)}\\
	\phi(k)=\frac{1}{\sqrt{2\pi}}\cdot\int_{-\infty}^\infty \Psi(x,0)\cdot
		\mathrm{e}^{-\mathrm{i}kx}\,\mathrm{d}x
\end{gather*}

Group velocity:
\[
	v_\mathrm{g}=\frac{\mathrm{d}\omega}{\mathrm{d}k}
\]

Phase velocity:
\[
	v_\mathrm{ph}=\frac{\omega}{k}
\]

Definition of bound and scattering states ($V(x\to\infty)=0$):
\begin{itemize}
	\item $E<0$: bound state
	\item $E>0$: scattering state
\end{itemize}

Delta function potential:
\begin{equation*}
	V(x)=-\alpha\delta(x)
\end{equation*}

Bound state:
\begin{gather*}
	\psi(x)=\frac{\sqrt{m\alpha}}{\hbar}\cdot\mathrm{e}^{-m\alpha|x|/\hbar^2}\\
	E=-\frac{m\alpha^2}{2\hbar^2}
\end{gather*}

Inner product:
\begin{gather*}
	\braket{\alpha|\beta}=a_1^* b_1 + a_2^* b_2 + \dots + a_N^* b_N\\
	\braket{\beta|\alpha}=\braket{\alpha|\beta}^*\\
	\braket{f|g}=\int_a^b f(x)^* g(x)\,\mathrm{d}x
\end{gather*}

Schwarz inequality:
\[
	\left|\braket{f|g}\right| \le \sqrt{\braket{f|f}\cdot\braket{g|g}}
\]

Orthonormal set of functions $\{f_n\}$:
\[
	\braket{f_m|f_n} = \delta_{mn}
\]

Complete set of functions $\{f_n\}$:
\begin{gather*}
	f(x)=\sum_{n=1}^\infty c_n\cdot f_n(x)\\
	c_n=\braket{f_n|f}
\end{gather*}

Expectation value of an observable:
\[
	\langle Q\rangle=\int \psi^*\hat Q\psi\,\mathrm{d}x=\braket{\psi|\hat Q|\psi}
\]

Hermitian operators:
\[
	\braket{f|\hat Q f}=\braket{\hat Qf|f}
\]

Properties of eigenfunctions of hermitian operators $\hat Q$ with discrete spectra:
\begin{enumerate}
	\item $\hat Q f=qf\,\curvearrowright\,q\in\mathbb{R}$
	\item $\hat Q f=qf,\,\, \hat Q g=q'g,\,\,q\neq q'\,\curvearrowright\,\braket{f|g}=0$
\end{enumerate}

Momentum space wave function:
\begin{gather*}
	\Phi(p,t)=\frac{1}{\sqrt{2\pi\hbar}}\cdot\int_{-\infty}^\infty \mathrm{e}^{-\mathrm{i}px/\hbar}
		\cdot\Psi(x, t)\,\mathrm{d}x\\
	\Psi(x,t)=\frac{1}{\sqrt{2\pi\hbar}}\cdot\int_{-\infty}^\infty \mathrm{e}^{\mathrm{i}px/\hbar}
		\cdot\Phi(p,t)\,\mathrm{d}p
\end{gather*}

Commutator of two operators $\hat A, \hat B$:
\[
	[\hat A, \hat B]:=\hat A\hat B - \hat B\hat A
\]

Anticommutator of two operators $\hat A, \hat B$:
\[
	\{\hat A, \hat B\}:=\hat A\hat B + \hat B\hat A
\]

Generalized uncertainty principle (observable $A$: $\sigma_A^2=\braket{\left(\hat A-\langle A\rangle\right)\Psi|\left(\hat A-\langle A\rangle\right)\Psi}$,\\ observable $B$: $\sigma_B^2=\braket{\left(\hat B-\langle B\rangle\right)\Psi|\left(\hat B-\langle B\rangle\right)\Psi}$):
\[
	\sigma_A^2\sigma_B^2\ge\left(\frac{1}{2\mathrm{i}} \langle [\hat A,\hat B]\rangle\right)^2
\]

Time dependence of an expectation value ($Q$: observable, $H$: Hamiltonian):
\begin{gather*}
	\frac{\mathrm{d}}{\mathrm{d}t}\langle Q\rangle=
		\frac{\mathrm{i}}{\hbar}\langle[\hat H,\hat Q]\rangle+\langle\frac{\partial\hat Q}{\partial t}\rangle
\end{gather*}

Schroedinger equation in three dimensions ($\vec p\to\frac{\hbar}{\mathrm{i}}\vec\nabla$):
\[
	\mathrm{i}\hbar\frac{\partial\Psi}{\partial t}=-\frac{\hbar^2}{2m}\Delta\Psi + V\Psi
\]

Normalization requirement:
\[
	\int_{\mathbb{R}^3}|\Psi|^2\,\mathrm{d}V=1
\]

Probability current and conservation of probability:
\begin{gather*}
	\vec j=\frac{\mathrm{i}\hbar}{2m}\cdot(\Psi\vec\nabla\Psi^* - \Psi^* \vec\nabla\Psi)\\
	\vec\nabla\cdot\vec j=-\frac{\partial}{\partial t}|\Psi|^2
\end{gather*}

Time-independent Schroedinger equation:
\[
	-\frac{\hbar^2}{2m}\Delta\psi + V\psi=E\psi
\]

General solution of the separable time-dependent Schroedinger equation:
\[
	\Psi(\vec r,t)=\sum_{n=1}^\infty c_n\psi_n(\vec r)\cdot\mathrm{e}^{-\mathrm{i}E_n\cdot t/\hbar}
\]

Canonical commutation relations ($i,j\in\{1,2,3\}$):
\begin{gather*}
	[r_i,p_j]=\mathrm{i}\hbar\delta_{ij}\\
	[r_i,r_j]=[p_i,p_j]=0
\end{gather*}

Legendre polynomials (Rodrigues formula):
\[
	P_l(x)=\frac{1}{2^l l!}\cdot\left(\frac{\mathrm{d}}{\mathrm{d}x}\right)^l \cdot(x^2-1)^l
\]

Associated Legendre functions:
\[
	P_l^m(x)=(1-x^2)^{|m|/2}\cdot\left(\frac{\mathrm{d}}{\mathrm{d}x}\right)^{|m|}\cdot P_l(x)
\]

Spherical harmonics (normalized angular wave function):
\begin{gather*}
	\text{Azimuthal quantum number: } l=0,1,2,\dots\\
	\text{Magnetic quantum number: } m=-l,\,-l+1,\,\dots,\,l-1,\,l\\
	Y_l^m(\vartheta,\varphi)=\sqrt{\frac{2l+1}{4\pi}\cdot\frac{(l-|m|)!}{(l+|m|)!}}
		\cdot\mathrm{e}^{\mathrm{i}m\varphi}\cdot P_l^m\left(\cos(\vartheta)\right)
		\cdot\begin{cases}(-1)^m, &m\ge 0\\ 1, &m<0\end{cases}
\end{gather*}

Orthonormality:
\[
	\int_0^{2\pi}\mathrm{d}\varphi\,\int_0^\pi\mathrm{d}\vartheta\,\sin(\vartheta)\,
		Y_l^m(\vartheta,\varphi)^*\cdot Y_{l'}^{m'}(\vartheta,\varphi)=\delta_{ll'}\delta_{mm'}
\]

Radial equation ($\psi(r,\vartheta,\varphi)=R(r)\cdot Y(\vartheta,\varphi)$, $u(r)\equiv r\cdot R(r)$):
\[
	-\frac{\hbar^2}{2m}\cdot\frac{\mathrm{d}^2 u}{\mathrm{d}r^2}
		+ \underbrace{\left[V+\frac{\hbar^2}{2m}\cdot\frac{l(l+1)}{r^2}\right]}_\text{effective potential}\cdot u = E\cdot u
\]

Radial wave equation for the hydrogen atom:
\[
	-\frac{\hbar^2}{2m_\mathrm{e}}\cdot\frac{\mathrm{d}^2 u}{\mathrm{d}r^2}
		+\left[-\frac{e^2}{4\pi\varepsilon_0}\cdot\frac{1}{r}+\frac{\hbar^2}{2}\cdot\frac{l(l+1)}{r^2}\right]\cdot u=E\cdot u
\]

Laguerre polynomial:
\[
	L_q(x)=\mathrm{e}^x\cdot\left(\frac{\mathrm{d}}{\mathrm{d}x}\right)^q\,\left(\mathrm{e}^{-x}\cdot x^q\right)
\]

Associated Laguerre polynomial:
\[
	L_{q-p}^p(x)=(-1)^p\cdot\left(\frac{\mathrm{d}}{\mathrm{d}x}\right)^p\,L_q(x)
\]

Energies (degeneracy of $E_n$: $n^2$):
\[
	E_n=-\left[\frac{m_\mathrm{e}}{2\hbar ^2}\cdot\left(\frac{e^2}{4\pi\varepsilon_0}\right)\right]\cdot\frac{1}{n^2},\quad n=1,2,\dots
\]

Bohr radius:
\[
	a_\mathrm{B}=\frac{4\pi\varepsilon_0\hbar^2}{m_\mathrm{e}e^2}
\]

Hydrogen wave function:
\begin{gather*}
	\psi_{nlm}(r,\vartheta,\varphi)=
		\sqrt{\left(\frac{2}{na}\right)^3\cdot\frac{(n-l-1)!}{2n\left[(n+l)!\right]^3}}
		\cdot\mathrm{e}^{-\nicefrac{r}{na}}\cdot\left(\frac{2r}{na}\right)^l
		\cdot\left[L_{n-l-1}^{2l+1}\left(\nicefrac{2r}{na}\right)\right]\cdot Y_l^m(\vartheta,\varphi)\\
	n=1,2,3,\,\dots;\,\,l=0,1,2,\,\dots,\,n-1;\,\,m=-l,\,-l+1,\,\dots,\,l-1,\,l
\end{gather*}

Angular momentum operator:
\[
	\hat{\vec L}=\hat{\vec r}\wedge\left(\frac{\mathrm{i}}{\hbar}\vec\nabla\right)
\]

Commutation relations ($L^2\equiv L_x^2+L_y^2+L_z^2$):
\begin{gather*}
	[L_x,L_y]=\mathrm{i}\hbar L_z;
		\quad[L_y,L_z]=\mathrm{i}\hbar L_x;
		\quad[L_z,L_x]=\mathrm{i}\hbar L_y\\
	[L^2,\vec L]=0
\end{gather*}

Eigenvalues ($Y_l^m$: spherical harmonics):
\begin{gather*}
	L^2\, Y_l^m=\hbar^2l(l+1)\,Y_l^m;\quad
		L_z\,Y_l^m=\hbar m\,Y_l^m\\
	l=0,\,\nicefrac{1}{2},\,1,\,\nicefrac{3}{2},\,\dots;\quad
		m=-l,-l+1,\dots,l-1,l
\end{gather*}

Commutation relations for spin:
\[
	[S_x,S_y]=\mathrm{i}\hbar S_z;
		\quad[S_y,S_z]=\mathrm{i}\hbar S_x;
		\quad[S_z,S_x]=\mathrm{i}\hbar S_y
\]

Eigenvectors and eigenvalues of $S^2$ and $S_z$:
\begin{gather*}
	S^2\ket{s\,m}=\hbar^2s(s+1)\,\ket{s\,m};
		\quad S_z\,\ket{s\,m}=\hbar m\ket{s\,m}\\
	s=0,\,\nicefrac{1}{2},\,1,\,\nicefrac{3}{2},\,\dots;
		\quad m=-s,-s+1,\dots,s-1,s
\end{gather*}

Eigenstates of spin $\nicefrac{1}{2}$ ($s=\nicefrac{1}{2}$):
\begin{itemize}
	\item Spin up: $\ket{\frac{1}{2}\,\frac{1}{2}} = \ket{\frac{1}{2}\,\uparrow}$
	\item Spin down: $\ket{\frac{1}{2}\,\left(-\frac{1}{2}\right)} = \ket{\frac{1}{2}\,\downarrow}$
\end{itemize}

Spinor:
\begin{gather*}
	\chi=\begin{bmatrix}a\\b\end{bmatrix}=a\chi_+ + b\chi_-\\
	\chi_+=\begin{bmatrix}1\\0\end{bmatrix},\,
	\chi_-=\begin{bmatrix}0\\1\end{bmatrix}\\
	|a|^2+|b|^2=1
\end{gather*}

Pauli matrices:
\[
	\sigma_x=\begin{bmatrix}0 & 1\\ 1 & 0\end{bmatrix},
	\quad\sigma_y=\begin{bmatrix}0 & -\mathrm{i}\\ \mathrm{i} & 0\end{bmatrix},
	\quad\sigma_z=\begin{bmatrix}1 & 0\\ 0 & -1\end{bmatrix}
\]

Spin operator ($\vec\sigma=[\sigma_x,\sigma_y,\sigma_z]$):
\[
	\vec S = \frac{\hbar}{2}\cdot\vec\sigma
\]

Hamiltonian of a charged particle at rest in a magnetic field ($\gamma$: gyromagnetic ratio):
\[
	H=-\gamma\vec B\cdot\vec S
\]

Triplet states ($s=1$):
\begin{enumerate}
	\item $\ket{1\,1} = \,\uparrow\,\uparrow$
	\item $\ket{1\,0} = \frac{1}{\sqrt{2}}\left(\uparrow\,\downarrow+\downarrow\,\uparrow\right)$
	\item $\ket{1\,(-1)} = \,\downarrow\,\downarrow$
\end{enumerate}

Singlet state ($s=0$):
\[
	\ket{0\,0} = \frac{1}{\sqrt{2}}\left(\uparrow\,\downarrow-\downarrow\,\uparrow\right)
\]

Addition of spins $s_1$ and $s_2$:
\[
	s=(s_1+s_2),\,(s_1+s_2-1),\,(s_1+s_2-2),\dots,|s_1-s_2|
\]

Combined states ($C_{m_1 m_2 m}^{s_1 s_2 s}$: Clebsch-Gordan coefficients):
\begin{gather*}
	\ket{s\,m}=\sum_{m_1+m_2=m} C_{m_1 m_2 m}^{s_1 s_2 s}\cdot\ket{s_1\,m_1}\,\ket{s_2\,m_2}\\
	\ket{s_1\,m_1}\,\ket{s_2\,m_2}=\sum_s C_{m_1 m_2 m}^{s_1 s_2 s}\cdot\ket{s\,m}
\end{gather*}

Symmetrization requirement (``$+$'': bosons, ``$-$'': fermions):
\[
	\psi(\vec r_1,\vec r_2)=\pm\psi(\vec r_2,\vec r_1)
\]
%
%
%
%
%
\section{CM}
\textbf{\Large{Drude's model}}

Avogadro's number:
\[
	N_\mathrm{A}=6.022\cdot 10^{23}\,\frac{1}{\mathrm{mol}}
\]

Mass density:
\[
	\varrho_\mathrm{m}=\frac{m}{V},\quad[\varrho_\mathrm{m}]=\frac{\mathrm{g}}{\mathrm{cm}^3}
\]

Moles per $\mathrm{cm}^3$ ($M$: molar mass, $[M]=\nicefrac{\mathrm{g}}{\mathrm{mol}}$):
\[
	\frac{\varrho_\mathrm{m}}{M}
\]

Conduction electron density ($Z$: number of valence electrons):
\[
	n=\frac{N}{V}=N_\mathrm{A}\cdot Z\cdot\frac{\varrho_\mathrm{m}}{M},\quad[n]=\frac{1}{\mathrm{cm}^3}
\]

Volume per conduction electron (modeled as a sphere):
\[
	\frac{V}{N}=\frac{1}{n}=\frac{4\pi}{3}\cdot r_\mathrm{s}^3\,\curvearrowright\,r_\mathrm{s}=\left(\frac{3}{4\pi}\cdot \frac{1}{n}\right)^{\nicefrac{1}{3}}
\]

Assumptions:
\begin{itemize}
	\item no $\mathrm{e}^-$--$\mathrm{e}^-$ interactions: \emph{independent electron} approximation
	\item no $\mathrm{e}^-$--ion interaction: \emph{free electron} approximation
	\item probability of a collision per unit time: $\nicefrac{1}{\tau}$ ($\tau$: \emph{relaxation time})
\end{itemize}

Resistivity ($\vec E$: electric field, $[E]=\nicefrac{\mathrm{V}}{\mathrm{m}}$, $\vec j$: current density, $[j]=\nicefrac{\mathrm{A}}{\mathrm{m}^2}$):
\[
	\vec E=\varrho\cdot\vec j,\quad[\varrho]=\Upomega\cdot\mathrm{m}
\]

Conductivity:
\[
	\sigma\equiv\frac{1}{\varrho},\quad[\sigma]=\frac{\mathrm{S}}{\mathrm{m}}\equiv\frac{\mathrm{mho}}{\mathrm{m}}\equiv\frac{\mho}{\mathrm{m}}
\]

Current density of $n$ electrons per unit volume moving with velocity $\vec v$:
\[
	\vec j=-ne\vec v
\]

Average electron velocity:
\[
	\vec v_\mathrm{average}=-\frac{1}{m_\mathrm{e}}\cdot e\tau\cdot\vec E\,\curvearrowright\,\vec j=\frac{ne^2\tau}{m_\mathrm{e}}\cdot\vec E
\]

Current density at time $t$:
\[
	\vec j=-ne\cdot\frac{\vec p(t)}{m_\mathrm{e}}
\]

Probability of a collision before $t+\mathrm{d}t$:
\[
	\frac{\mathrm{d}t}{\tau}
\]

Probability of no collision before $t+\mathrm{d}t$:
\[
	1-\frac{\mathrm{d}t}{\tau}
\]

Additional momentum due to external force $\vec F(t)$:
\[
	\mathrm{d}\vec p=\vec F(t)\,\mathrm{d}t+\mathcal{O}\left((\mathrm{d}t)^2\right)
\]

Contribution of electrons without any collision during $t+\mathrm{d}t$ to the average momentum (correction due to electrons with a collision before $t+\mathrm{d}t$: $\mathcal{O}\left((\mathrm{d}t)^2\right)$):
\begin{gather*}
	\vec p(t+\mathrm{d}t)=\left(1-\frac{\mathrm{d}t}{\tau}\right)\cdot\left[\vec p(t)+\vec F(t)\,\mathrm{d}t+\mathcal{O}\left((\mathrm{d}t)^2\right)\right]\\
	\frac{\vec p(t+\mathrm{d}t)-\vec p(t)}{\mathrm{d}t}=-\frac{1}{\tau}\cdot\vec p(t)+\vec F(t)\\
	\curvearrowright\,\dot{\vec{p}}(t)=-\frac{1}{\tau}\cdot\vec p(t)+\vec F(t)\quad\text{(e.o.m)}
\end{gather*}

\textbf{\normalsize{AC electrical conductivity}}

\[
	\vec E(t)=\Re\left\{\vec E(\omega)\cdot\mathrm{e}^{-\mathrm{i}\omega t}\right\}
\]

e.o.m:
\[
	\dot{\vec{p}}=-\frac{1}{\tau}\cdot\vec p-e\cdot\vec E
\]

Steady state solution:
\[
	\vec p(t)=\Re\left\{\vec p(\omega)\cdot\mathrm{e}^{-\mathrm{i}\omega t}\right\}
\]

\begin{gather*}
	\curvearrowright\,-\mathrm{i}\omega\cdot\vec p(\omega)=-\frac{1}{\tau}\cdot\vec p(\omega)-e\cdot\vec E(\omega)\quad\curvearrowright\,\vec p(\omega)=-\frac{e\cdot\vec E}{\nicefrac{1}{\tau}-\mathrm{i}\omega}\\
	\vec j(t)=\Re\left\{\vec j(\omega)\cdot\mathrm{e}^{-\mathrm{i}\omega t}\right\}=\Re\left\{-\frac{ne}{m_\mathrm{e}}\cdot\vec p(\omega)\cdot\mathrm{e}^{-\mathrm{i}\omega t}\right\}\\
	\curvearrowright\,\vec j(\omega)=-\frac{ne}{m_\mathrm{e}}\cdot\vec p(\omega)=-\frac{ne}{m_\mathrm{e}}\cdot\left(-\frac{e\cdot\vec E(\omega)}{\nicefrac{1}{\tau}-\mathrm{i}\omega}\right)=\frac{ne^2}{m_\mathrm{e}}\cdot\frac{\vec E(\omega)}{\nicefrac{1}{\tau}-\mathrm{i}\omega}\\
	\curvearrowright\,\vec j(\omega)=\sigma(\omega)\cdot\vec E(\omega),\,\,\,\sigma(\omega)=\frac{\sigma_\mathrm{DC}}{1-\mathrm{i}\omega\tau},\,\,\,\sigma_\mathrm{DC}=\frac{ne^2\tau}{m_\mathrm{e}}
\end{gather*}

$\lambda\gg\text{ electronic mean free path}$:
\[
	\curvearrowright\,\vec j(\vec r,\omega)=\sigma(\omega)\cdot\vec E(\vec r,\omega)
\]

Maxwell's equations (cgs units):
\begin{gather*}
	\vec\nabla\cdot\vec E=0\\
	\vec\nabla\cdot\vec H=0\\
	\vec\nabla\wedge\vec E=-\frac{1}{c}\cdot\frac{\partial\vec H}{\partial t}\\
	\vec\nabla\wedge\vec H=\frac{1}{c}\cdot\left[4\pi\vec j+\frac{\partial\vec E}{\partial t}\right]
\end{gather*}

\begin{gather*}
	\vec\nabla\wedge(\vec\nabla\wedge\vec E)=\vec\nabla(\vec\nabla\cdot\vec E)=\vec\nabla(\underbrace{\vec\nabla\cdot\vec E}_{=0})-\vec\nabla^2\vec E=-\frac{1}{c}\cdot\frac{\partial}{\partial t}(\vec\nabla\wedge\vec H)=-\frac{1}{c}\,(-\mathrm{i}\omega)\cdot(\vec\nabla\wedge\vec H)\\
	=\frac{\mathrm{i}\omega}{c}\cdot(\vec\nabla\wedge\vec H)=\frac{\mathrm{i}\omega}{c^2}\cdot\left[4\pi\sigma(\omega)\cdot\vec E(\omega)-\mathrm{i}\omega\cdot\vec E(\omega)\right]\\
	\curvearrowright\,-\vec\nabla^2\vec E=\vec E\cdot\left(\frac{\mathrm{i}\omega}{c^2}\cdot 4\pi\sigma(\omega)+\frac{\omega^2}{c^2}\right)=\left(1+\frac{4\pi\mathrm{i}\sigma(\omega)}{\omega}\right)\cdot\frac{\omega^2}{c^2}\cdot\vec E
\end{gather*}

Complex dielectric constant:
\[
	\varepsilon(\omega)\equiv 1+\frac{4\pi\mathrm{i}\sigma(\omega)}{\omega}
\]

\[
	\curvearrowright\,-\vec\nabla^2\vec E=\varepsilon(\omega)\cdot\frac{\omega^2}{c^2}\cdot\vec E
\]

For $\omega\tau\gg 1$:
\begin{gather*}
	\varepsilon(\omega)=1+\frac{4\pi\mathrm{i}}{\omega}\cdot\sigma_\mathrm{DC}\cdot\frac{1}{\underbrace{1-\mathrm{i}\omega\tau}_{\approx-\mathrm{i}\omega\tau}}\approx 1-\frac{4\pi}{\omega^2\tau}\cdot\sigma_\mathrm{DC}=1-\frac{1}{\omega^2}\cdot\frac{4\pi ne^2}{m_\mathrm{e}}\\
	=1-\frac{\omega_\mathrm{p}^2}{\omega^2}
\end{gather*}

\emph{Plasma frequency}:
\[
	\omega_\mathrm{p}^2=\frac{4\pi ne^2}{m_\mathrm{e}}
\]

\begin{itemize}
	\item $\omega<\omega_\mathrm{p}$: $\varepsilon<0$: no propagation of e.m. waves
	\item $\omega>\omega_\mathrm{p}$: $\varepsilon>0$: propagation of e.m waves is possible
\end{itemize}

\textbf{\normalsize{Thermal conductivity}}

Wiedemann-Franz law ($\kappa$: thermal conductivity, $[\kappa]=\nicefrac{\mathrm{W}}{\mathrm{m}\cdot\mathrm{K}}$; $\sigma$: electrical conductivity, $[\sigma]=\nicefrac{1}{\Upomega\cdot\mathrm{m}}$; $T$: temperature, $[T]=\mathrm{K}$):
\[
	\frac{\kappa}{\sigma}\propto T
\]

Fourier's law ($\vec j^\mathrm{q}$: thermal conductivity, $[j^\mathrm{q}]=\nicefrac{\mathrm{W}}{\mathrm{m}^2}$; $\vec\nabla T$: temperature gradient,\\ $[\nabla T]=\nicefrac{\mathrm{K}}{\mathrm{m}}$):
\[
	\vec j^\mathrm{q}=-\kappa\cdot\vec\nabla T
\]

\textbf{\Large{Sommerfeld's model}}

Maxwell-Boltzmann distribution:
\[
	f_\mathrm{MB}(\vec v)=n\cdot\left(\frac{m}{2\pi k_\mathrm{B}T}\right)^{\nicefrac{3}{2}}\cdot\exp\left\{-\frac{1}{2}\cdot\frac{mv^2}{k_\mathrm{B}T}\right\}
\]

\textbf{\normalsize{Ground state ($T\equiv 0$)}}

Time independent Schroedinger equation for one free and independent electron:
\[
	H\,\Psi(\vec r)=E\,\Psi(\vec r)\,\curvearrowright\,-\frac{\hbar^2}{2m_\mathrm{e}}\,\vec\nabla^2\Psi(\vec r)=E\,\Psi(\vec r)
\]

Born--von Karman/periodic boundary conditions for an electron confined to a cube of volume $V=L^3$:
\begin{gather*}
	\Psi(x+L,y,z)=\Psi(x,y,z)\\
	\Psi(x,y+L,z)=\Psi(x,y,z)\\
	\Psi(x,y,z+L)=\Psi(x,y,z)
\end{gather*}

Trial solution ignoring the boundary condition:
\begin{gather*}
	\Psi_{\vec k}(\vec r)=\frac{1}{\sqrt{V}}\cdot\mathrm{e}^{\mathrm{i}\vec k\cdot\vec r}\\
	\curvearrowright\,-\frac{\hbar^2}{2m}\,(\partial_x^2+\partial_y^2+\partial_z^2)\,\mathrm{e}^{\mathrm{i}\vec k\cdot\vec r}=\frac{\hbar^2}{2m}\cdot(k_x^2+k_y^2+k_z^2)\cdot\mathrm{e}^{\mathrm{i}\vec k\cdot\vec r}=\frac{\hbar^2|\vec k|^2}{2m}\cdot\mathrm{e}^{\mathrm{i}\vec k\cdot\vec r}\\
	\curvearrowright\,E(\vec k)=\frac{\hbar^2\vec k^2}{2m}\\
	\int_V\mathrm{d}^3x\,\left|\frac{1}{\sqrt{V}}\cdot\mathrm{e}^{\mathrm{i}\vec k\cdot\vec r}\right|^2=\frac{1}{V}\cdot\int_V\mathrm{d}^3x=1
\end{gather*}

$\Psi_{\vec k}(\vec r)$: eigenstate of the momentum operator:
\begin{gather*}
	\hat{\vec{p}}\equiv\frac{\hbar}{\mathrm{i}}\,\vec\nabla\\
	\curvearrowright\,\hat{\vec{p}}\,\mathrm{e}^{\mathrm{i}\vec k\cdot\vec r}=\frac{\hbar}{\mathrm{i}}\cdot\mathrm{i}\vec k\cdot\mathrm{e}^{\mathrm{i}\vec k\cdot r}=\hbar\vec k\cdot\mathrm{e}^{\mathrm{i}\vec k\cdot\vec r}
\end{gather*}

Momentum and velocity of an electron in level $\Psi_{\vec k}(\vec r)$:
\[
	\vec p=\hbar\vec k\quad\vec v=\frac{\hbar\vec k}{m_\mathrm{e}}
\]

de Broglie wavelength ($\vec k$: wave vector, $k\equiv|\vec k|$: wave number):
\[
	\lambda=\frac{2\pi}{k}
\]

Applying the boundary conditions:
\begin{gather*}
	\mathrm{e}^{\mathrm{i}k_x\cdot L}=\mathrm{e}^{\mathrm{i}k_y\cdot L}=\mathrm{e}^{\mathrm{i}k_z\cdot L}=1\\
	\curvearrowright\,k_x=\frac{2\pi n_x}{L},k_y=\frac{2\pi n_y}{L},k_z=\frac{2\pi n_z}{L}\quad n_x,n_y,n_z\in\mathbb{Z}
\end{gather*}

Number of allowed states in $k$-space of volume $\Omega$ ($V=L^3$):
\[
	\frac{\Omega}{\underbrace{(2\pi/L)^3}_\text{Volume of one state}}=\frac{\Omega\cdot V}{8\pi^3}
\]

Number of allowed $\vec k$ values within the \emph{Fermi sphere} (radius $k_\mathrm{F}$):
\[
	\frac{V}{8\pi^3}\cdot\Omega=\frac{V}{8\pi^3}\cdot\frac{4\pi}{3}\,k_\mathrm{F}^3=\frac{V}{6\pi^2}\cdot k_\mathrm{F}^3
\]

$N$ electrons with spin $\pm\frac{\hbar}{2}$:
\[
	N=2\cdot\frac{V}{6\pi^2}\cdot k_\mathrm{F}^3=\frac{V}{3\pi^2}\cdot k_\mathrm{F}^3
\]

Ground state of a system with $N$ electrons (density $n=\nicefrac{N}{V}$, $k_\mathrm{F}$: Fermi wave vector):
\[
	n=\frac{k_\mathrm{F}^3}{3\pi^2}
\]

\emph{Fermi momentum} and \emph{Fermi velocity}:
\[
	p_\mathrm{F}=\hbar k_\mathrm{F},\,v_\mathrm{F}=\frac{p_\mathrm{F}}{m_\mathrm{e}}
\]

\emph{Fermi energy}:
\[
	E_\mathrm{F}=\frac{\hbar^2 k_\mathrm{F}^2}{2m_\mathrm{e}}
\]

Ground state energy of $N$ electrons:
\[
	E=\underbrace{2}_{\text{Spin }\pm\nicefrac{\hbar}{2}}\cdot\sum_{k<k_\mathrm{F}}\frac{\hbar^2 k^2}{2m_\mathrm{e}}
\]

\begin{gather*}
	\sum_{\vec k}F(\vec k)=\left(\frac{V}{8\pi^3}\right)\cdot\sum_{\vec k}F(\vec k)\cdot\left(\frac{8\pi^3}{V}\right)\\
	\curvearrowright\,\lim_{V\to\infty}\,\frac{1}{V}\cdot\sum_{\vec k}F(\vec k)=\int\frac{\mathrm{d}\vec k}{8\pi^3}\,F(\vec k)\\
	\curvearrowright\,\underbrace{\frac{E}{V}}_\text{Energy density}=2\cdot\int_{k<k_\mathrm{F}}\frac{\mathrm{d}^3 k}{8\pi^3}\,\frac{\hbar^2 k^2}{2m_\mathrm{e}}=\frac{\hbar^2}{8m_\mathrm{e}\pi^3}\cdot\int_0^\pi\mathrm{d}\vartheta\,\sin(\vartheta)\,\int_0^{2\pi}\mathrm{d}\varphi\,\int_0^{k_\mathrm{F}}\mathrm{d}r r^4\\
	=\frac{\hbar^2}{8m_\mathrm{e}\pi^3}\cdot 2\cdot 2\pi\cdot\frac{1}{5}k_\mathrm{F}^5=\frac{\hbar^2 k_\mathrm{F}^5}{10\pi^2 m_\mathrm{e}}
\end{gather*}

Energy per electron:
\[
	\frac{E}{N}=\frac{E}{V}\cdot\frac{3\pi^2}{k_\mathrm{F}^3}=\frac{3}{10}\cdot\frac{\hbar^2 k_\mathrm{F}^2}{m_\mathrm{e}}=\frac{3}{5}\cdot E_\mathrm{F}
\]

\emph{Fermi temperature}:
\[
	T_\mathrm{F}\equiv\frac{E_\mathrm{F}}{k_\mathrm{B}}=\frac{\hbar^2 k_\mathrm{F}^2}{2k_\mathrm{B} m_\mathrm{e}}
\]

\textbf{\Large{Crystal lattices}}

\textbf{\normalsize{Bravais lattice}}

Position vectors in a Bravais lattice ($\vec a_i$: \emph{primitive vectors}):
\[
	\vec R=n_1\,\vec a_1+n_2\,\vec a_2+n_3\,\vec a_3;\quad n_1,n_2,n_3\in\mathbb{Z}
\]

\emph{Nearest neighbours}:
\begin{quote}
	Points in a Bravais lattice closest to a given point $\equiv$ \emph{coordination number}
\end{quote}

\emph{sc}/\emph{simple cubic} ($\hat x,\hat y,\hat z$: orthogonal unit vectors):
\begin{gather*}
	\vec a_1=a\,\hat x,\,\vec a_2=a\,\hat y,\,\vec a_3=a\,\hat z\\
	\text{coordination number: }6
\end{gather*}

\emph{bcc}/\emph{body-centered cubic}:
\begin{gather*}
	\vec a_1=a\,\hat x,\,\vec a_2=a\,\hat y,\,\vec a_3=\frac{a}{2}\,(\hat x+\hat y+\hat z)\\
	\vec a_1=\frac{a}{2}\,(\hat y+\hat z-\hat x),\,\vec a_2=\frac{a}{2}\,(\hat z+\hat x-\hat y),\,\vec a_3=\frac{a}{2}\,(\hat x+\hat y-\hat z)\\
	\text{coordination number: }8
\end{gather*}

\emph{fcc}/\emph{face-centered cubic}:
\begin{gather*}
	\vec a_1=\frac{a}{2}\,(\hat y+\hat z),\,\vec a_2=\frac{a}{2}\,(\hat z+\hat x),\,\vec a_3=\frac{a}{2}\,(\hat x+\hat y)\\
	\text{coordination number: }12
\end{gather*}

\emph{Primitive (unit) cell} of a lattice:
\begin{quote}
	Fills all of space without voids or overlap when translated along the primitive vectors $\vec a_i$; contains exactly one point of the lattice.\\
	$\vec r=x_1\,\vec a_1+x_2\,\vec a_2+x_3\,\vec a_3,\,x_i\in[0,1]$
\end{quote}

\textbf{\normalsize{Reciprocal lattice}}

Definition of the \emph{reciprocal} lattice ($\vec R$: Bravais lattice, $\vec K$: set of wave vectors):
\begin{gather*}
	\mathrm{e}^{\mathrm{i}\vec K\cdot(\vec r+\vec R)}\stackrel{!}{=}\mathrm{e}^{\mathrm{i}\vec K\cdot\vec r}\quad\forall\vec r,\vec R\\
	\curvearrowright\,\mathrm{e}^{\mathrm{i}\vec K\cdot\vec R}=1\quad\forall\vec R
\end{gather*}

Primitive vectors spanning the reciprocal lattice ($\vec a_i$: primitive vectors generating the \emph{direct lattice}):
\begin{gather*}
	\vec b_1=2\pi\,\frac{\vec a_2\wedge\vec a_3}{\vec a_1\cdot(\vec a_2\wedge\vec a_3)}\\
	\vec b_2=2\pi\,\frac{\vec a_3\wedge\vec a_1}{\vec a_1\cdot(\vec a_2\wedge\vec a_3)}=2\pi\,\frac{\vec a_3\wedge\vec a_1}{\vec a_2\cdot(\vec a_3\wedge\vec a_1)}\\
	\vec b_3=2\pi\,\frac{\vec a_1\wedge\vec a_2}{\vec a_1\cdot(\vec a_2\wedge\vec a_3)}=2\pi\,\frac{\vec a_1\wedge\vec a_2}{\vec a_3\cdot(\vec a_1\wedge\vec a_2)}\\
	\curvearrowright\,\vec b_i\cdot\vec a_j=2\pi\,\delta_{ij}\\
	\vec k=k_1\,\vec b_1+k_2\,\vec b_2+k_3\,\vec b_3\\
	\vec R=n_1\,\vec a_1+n_2\,\vec a_2+n_3\,\vec a_3\quad n_i\in\mathbb{Z}\\
	\curvearrowright\,\vec k\cdot\vec R=2\pi\,(k1\,n_1+k_2\,n_2+k_3\,n_3)\stackrel{!}{=}m\cdot 2\pi\quad m\in\mathbb{Z}\\
	\curvearrowright\,k_i\in\mathbb{Z}
\end{gather*}

Simple cubic (sc):
\begin{gather*}
	\vec a_1=a\,\hat x,\,\vec a_2=a\,\hat y,\,\vec a_3=a\,\hat z\\
	\vec b_1=2\pi\,\frac{a^2\,(\hat y\wedge\hat z)}{a^3\,\hat x\cdot(\hat y\wedge\hat z)}=\frac{2\pi}{a}\,\hat x\\
	\vec b_2=\frac{2\pi}{a}\,(\hat z\wedge\hat x)=\frac{2\pi}{a}\,\hat y\\
	\vec b_3=\frac{2\pi}{a}\,(\hat x\wedge\hat y)=\frac{2\pi}{a}\,\hat z\\
	\curvearrowright\,\text{Reciprocal lattice: sc}
\end{gather*}

Face-centered cubic (fcc):
\begin{gather*}
	\vec a_1=\frac{a}{2}\,(\hat y+\hat z),\,\vec a_2=\frac{a}{2}\,(\hat z+\hat x),\,\vec a_3=\frac{a}{2}\,(\hat x+\hat y)\\
	\vec b_1=2\pi\,\frac{(a/2)^2\,(\hat z+\hat x)\wedge(\hat x+\hat y)}{(a/2)^3\,(\hat y+\hat z)\cdot\left[(\hat z+\hat x)\wedge(\hat x+\hat y)\right]}=\frac{4\pi}{a}\,\frac{\hat y-\hat x+\hat z}{1+1}=\frac{4\pi}{a}\,\frac{1}{2}\,(\hat y+\hat z-\hat x)\\
	\vec b_2=\frac{4\pi}{a}\,\frac{1}{2}\,\left[(\hat x+\hat y)\wedge(\hat y+\hat z)\right]=\frac{4\pi}{a}\,\frac{1}{2}\,(\hat x+\hat z-\hat y)\\
	\vec b_3=\frac{4\pi}{a}\,\frac{1}{2}\,\left[(\hat y+\hat z)\wedge(\hat z+\hat x)\right]=\frac{4\pi}{a}\,\frac{1}{2}\,(\hat x+\hat y-\hat z)\\
	\curvearrowright\,\text{Reciprocal lattice: bcc}
\end{gather*}

Body-centered cubic (bcc):
\[
	\curvearrowright\,\text{Reciprocal lattice: fcc}
\]

Volume of the reciprocal lattice primitive cell ($v$: volume of the primitive cell of the direct lattice):
\[
	\frac{(2\pi)^3}{v}
\]

\emph{First Brillouin zone}:
\begin{quote}
	Wigner-Seitz primitive cell of the reciprocal lattice.
\end{quote}

\textbf{\normalsize{Lattice planes}}

\emph{Family of lattice planes}:
\begin{quote}
	Set of parallel, equally spaces (distance $d$) lattice planes (contain all points of the Bravais lattice).
\end{quote}

Classification of lattice planes:
\begin{itemize}
	\item For any family of lattice planes, there exists a set of reciprocal lattice vectors with\\ length $\nicefrac{2\pi}{d}$.
	\item For any reciprocal lattice vector $\vec K, |\vec K|=\nicefrac{2\pi}{d}$, there exists a family of lattice planes normal to $\vec K$ seperated by distance $d$.
\end{itemize}

Plane with \emph{Miller indices} $h,k,l$:
\begin{quote}
	Plane normal to the reciprocal lattice vector $\vec k=h\,\vec b_1+k\,\vec b_2+l\,\vec b_3$ ($h,k,l\in\mathbb{Z}$); $|\vec k|\stackrel{!}{=}\text{min}$.
\end{quote}
%
%
%
%
%
\section{P}

Lagrangian ($T$: kinetic energy, $V$: potential energy):
\[
	L\equiv T-V
\]

Action:
\[
	S[x]=\int_{t_a}^{t_b} L(x,\dot x,t)\,\mathrm{d}t\stackrel{!}{=}\text{min}
\]

Variation of the action:
\[
	\delta S=S[x+\delta x]-S[x]\stackrel{!}{=}0
\]

Endpoints of the extremum path $x$ shall be fixed:
\[
	\delta x(t_a)=\delta x(t_b)=0
\]

\begin{gather*}
	\curvearrowright\, S[x+\delta x]=\int_{t_a}^{t_b} L(x+\delta x,\dot x+\delta\dot x,t)\,\mathrm{d}t=\int_{t_a}^{t_b} \left[L(x,\dot x,t)+\frac{\partial L}{\partial x}\,\delta x+\frac{\partial L}{\partial \dot x}\,\delta\dot x\right]\,\mathrm{d}t\\
	=S[x]+\int_{t_a}^{t_b} \left[\frac{\partial L}{\partial x}\,\delta x+\frac{\partial L}{\partial\dot x}\,\delta\dot x\right]\,\mathrm{d}t\\
	\curvearrowright\,\delta S=\int_{t_a}^{t_b} \left[\frac{\partial L}{\partial x}\,\delta x+\frac{\partial L}{\partial\dot x}\,\frac{\mathrm{d}}{\mathrm{d}t}\,\delta x\right]\,\mathrm{d}t=\int_{t_a}^{t_b} \frac{\partial L}{\partial x}\,\delta x\,\mathrm{d}t+\underbrace{\left.\frac{\partial L}{\partial\dot x}\,\delta x\right|_{t_a}^{t_b}}_{=0}-\int_{t_a}^{t_b} \frac{\mathrm{d}}{\mathrm{d}t}\left(\frac{\partial L}{\partial\dot x}\right)\,\delta x\,\mathrm{d}t\\
	=\int_{t_a}^{t_b}\delta x\,\left[\frac{\partial L}{\partial x}-\frac{\mathrm{d}}{\mathrm{d}t}\,\frac{\partial L}{\partial \dot x}\right]\,\mathrm{d}t\stackrel{!}{=}0\quad\text{for arbitrary }\delta x\\
	\curvearrowright\,\frac{\mathrm{d}}{\mathrm{d}t}\,\frac{\partial L}{\partial\dot x}-\frac{\partial L}{\partial x}=0
\end{gather*}

Euler-Lagrange equation(s) for $n$ coordinates:
\[
	\frac{\mathrm{d}}{\mathrm{d}t}\,\frac{\partial L}{\partial \dot q_i}-\frac{\partial L}{\partial q_i}=0,\quad i=1,\dots,n
\]

Problem 2--1: Free particle with Lagrangian $L=\frac{1}{2}m\dot x^2$. Show:
\[
	S=\frac{m}{2}\,\frac{(x_b-x_a)^2}{t_b-t_a}
\]

Solution:
\begin{gather*}
	x(t)=x_a+\frac{x_b-x_a}{t_b-t_a}\cdot (t-t_a)\\
	\curvearrowright\,\dot x(t)=\frac{x_b-x_a}{t_b-t_a}\\
	\curvearrowright\,S=\frac{m}{2}\,\int_{t_a}^{t_b}\,\frac{(x_b-x_a)^2}{(t_b-t_a)^2}\,\mathrm{d}t=\frac{m}{2}\,\left.\frac{(x_b-x_a)^2}{(t_b-t_a)^2}\cdot t\right|_{t_a}^{t_b}=\frac{m}{2}\,\frac{(x_b-x_a)^2}{t_b-t_a}
\end{gather*}

Problem 2--2: Harmonic oscillator with Lagrangian $L=\frac{m}{2}(\dot x^2-\omega^2 x^2)$, $T\equiv t_b-t_a$. Show:
\[
	S=\frac{m\omega}{2\sin(\omega T)}\,\left[(x_a^2+x_b^2)\,\cos(\omega T)-2x_a x_b\right]
\]

Solution:
\begin{gather*}
	x(t)=A\sin(\omega t)+B\cos(\omega t)\\
	x(t_a\equiv 0)=B=x_a\\
	x(t_b-t_a\equiv T)=A\sin(\omega T)+x_a\cos(\omega T)=x_b\\
	\curvearrowright\,A=\frac{x_b-x_a\cos(\omega T)}{\sin(\omega T)}\\
	\curvearrowright\,x(t)=\frac{x_b-x_a\cos(\omega T)}{\sin(\omega T)}\,\sin(\omega t)+x_a\cos(\omega t)\\
	\dot x(t)=A\omega\cos(\omega t)-x_a\omega\sin(\omega t)\\
	S=\frac{m}{2}\,\int_{t_a}^{t_b} \dot x\dot x\,\mathrm{d}t-\frac{m\omega^2}{2}\,\int_{t_a}^{t_b} x^2\,\mathrm{d}t=\left.\frac{m}{2}\,\dot x x\right|_{t_a}^{t_b}-\frac{m}{2}\,\int_{t_a}^{t_b}\underbrace{\ddot x}_{-\omega^2 x} x\,\mathrm{d}t-\frac{m\omega^2}{2}\,\int_{t_a}^{t_b} x^2\,\mathrm{d}t\\=\left.\frac{m}{2}\,x\dot x\right|_{t_a}^{t_b}\\
	\curvearrowright\,S=\frac{m}{2}\,\left[x_b\dot x(T)-x_a\dot x(0)\right]=\frac{m}{2}\,\left[x_b A\omega\cos(\omega T)-x_b x_a\omega\sin(\omega T)-x_a A\omega\right]\\
	=\frac{m\omega}{2\sin(\omega T)}\,\left[x_b\left(x_b-x_a\cos(\omega T)\right)\cos(\omega T)-x_b x_a\sin^2(\omega T)-x_a\left(x_b-x_a\cos(\omega T)\right)\right]\\
	=\dots=\frac{m\omega}{2\sin(\omega T)}\,\left[(x_a^2+x_b^2)\,\cos(\omega T)-2x_a x_b\right]
\end{gather*}

Problem 2--3: Particle under constant force $F$ with $L=\frac{1}{2}m\dot x^2+Fx$, $T\equiv t_b-t_a$. Show:
\[
	S=\frac{m(x_b-x_a)^2}{2T}+\frac{FT(x_b+x_a)}{2}-\frac{F^2 T^3}{24 m}
\]

Solution:
\begin{gather*}
	x(t)=x_a+\dot x(t_a)\cdot(t-t_a)+\frac{1}{2}\frac{F}{m}\cdot(t-t_a)^2\\
	x(t_b)=x_b=x_a+\dot x(t_a)T+\frac{F}{2m}T^2\\
	\curvearrowright\,\dot x(t_a)=\frac{x_b-x_a}{T}-\frac{FT}{2m}\\
	\curvearrowright\,x(t)=x_a+\left(\frac{x_b-x_a}{T}-\frac{FT}{2m}\right)\cdot(t-t_a)+\frac{F}{2m}\cdot(t-t_a)^2\\
	\frac{m}{2}\dot x^2=\frac{m}{2}\left(\frac{x_b-x_a}{T}-\frac{FT}{2m}\right)^2+\left(\frac{x_b-x_a}{T}-\frac{FT}{2m}\right)F\cdot(t-t_a)+\frac{F^2}{2m}\cdot(t-t_a)^2\\
	Fx=Fx_a+\left(\frac{x_b-x_a}{T}-\frac{FT}{2m}\right)F\cdot(t-t_a)+\frac{F^2}{2m}\cdot(t-t_a)^2\\
	\frac{m}{2}\dot x^2+Fx=\frac{F^2}{m}\cdot(t-t_a)^2+2F\left(\frac{x_b-x_a}{T}-\frac{FT}{2m}\right)\cdot(t-t_a)+Fx_a+\frac{m}{2}\left(\frac{x_b-x_a}{T}-\frac{FT}{2m}\right)^2\\
	\curvearrowright\,S=\int_{t_a}^{t_b}\,\left(\frac{m}{2}\dot x^2+Fx\right)\,\mathrm{d}t=\frac{F^2 T^3}{3m}+F\left(\frac{x_b-x_a}{T}-\frac{FT}{2m}\right)T^2+Fx_a T+\frac{mT}{2}\left(\frac{x_b-x_a}{T}-\frac{FT}{2m}\right)^2\\
	=\dots=\frac{m}{2T}(x_b-x_a)^2+\frac{FT}{2}(x_b+x_a)-\frac{F^2 T^3}{24m}
\end{gather*}

%Problem 2--4: Momentum: $p\equiv\frac{\partial L}{\partial\dot x}$. Show:
%\[
	%\left.\frac{\partial L}{\partial\dot x}\right|_{x=x_b}=\frac{\partial S}{\partial x_b};\quad\left.\frac{\partial L}{\partial\dot x}\right|_{x=x_a}=-\frac{\partial S}{\partial x_a}
%\]

Infinitesimal transformation ($\varepsilon\to 0$):
\begin{gather*}
	q\,\to\,q+\varepsilon\delta\\
	\curvearrowright\,L(q,\dot q)\,\to\,L(q+\varepsilon\delta q,\dot q+\varepsilon\delta\dot q)=L(q,\dot q)+\varepsilon\delta q\,\frac{\partial L}{\partial q}+\varepsilon\delta\dot q\,\frac{\partial L}{\partial\dot q}\\=L(q,\dot q)+\varepsilon\delta q\,\frac{\mathrm{d}}{\mathrm{d}t}\,\frac{\partial L}{\partial\dot q}+\varepsilon\frac{\partial L}{\partial\dot q}\,\frac{\mathrm{d}}{\mathrm{d}t}\,\delta q
	=L(q,\dot q)+\frac{\mathrm{d}}{\mathrm{d}t}\,\left(\varepsilon\delta q\,\frac{\partial L}{\partial\dot q}\right)\\
	\curvearrowright\,\delta L=\frac{\mathrm{d}}{\mathrm{d}t}\,\left(\frac{\partial L}{\partial\dot q}\,\varepsilon\delta q\right)
\end{gather*}

Noether current:
\[
	j\equiv\frac{\partial L}{\partial\dot q}\,\delta q
\]

Hamiltonian ($p\equiv\frac{\partial L}{\partial\dot q}$: momentum):
\[
	H\equiv T+V=p\dot q-L
\]

Lorentz transformation along the $x$-axis ($\gamma\equiv\frac{1}{\sqrt{1-\beta^2}}$, $\beta\equiv\frac{v}{c}$):
\begin{enumerate}
	\item $x'=\gamma(x-vt)$
	\item $y'=y$
	\item $z'=z$
	\item $t'=\gamma(t-\frac{v}{c^2}x)$
\end{enumerate}

Position four-vector:
\[
	x^0\equiv ct,\,x^1=x,\,x^2=y,\,x^3=z
\]

Lorentz transformation using four-vectors:
\begin{enumerate}
	\item $x'^0=\gamma(x^0-\beta x^1)$
	\item $x'^1=\gamma(x^1-\beta x^0)$
	\item $x'^2=x^2$
	\item $x'^3=x^3$
\end{enumerate}

Matrix form:
\begin{gather*}
	\begin{bmatrix}x'^0\\x'^1\\x'^2\\x'^3\end{bmatrix}=\begin{bmatrix}\gamma & -\gamma\beta & 0 & 0\\ -\gamma\beta & \gamma & 0 & 0\\ 0 & 0 & 1 & 0\\ 0 & 0 & 0 & 1\\\end{bmatrix}\cdot\begin{bmatrix}x^0\\x^1\\x^2\\x^3\end{bmatrix}\\
	\curvearrowright\,x'^\mu=\sum_{\nu=0}^3 \Lambda^\mu_\nu x^\nu\quad\mu:\text{ row},\,\,\nu:\text{ column}
\end{gather*}

Definition of a four-vector ($\Lambda$: Lorentz transformation):
\[
	a'^\mu=\Lambda^\mu_\nu x^\nu
\]

Contravariant four-vector:
\[
	a^\mu=(a^0,a^1,a^2,a^3)^t
\]

Covariant four-vector:
\[
	a_\mu=(-a^0,a^1,a^2,a^3)
\]

Scalar product of two four-vectors (invariant under Lorentz transformations):
\[
	a\cdot b=a_\mu b^\mu=a^\mu b_\mu=-a^0 b^0+a^1 b^1+a^2 b^2+a^3 b^3
\]

Minkowski metric $(-,+,+,+)$:
\begin{gather*}
	g_{\mu\nu}=g^{\mu\nu}=\diag(-1,1,1,1)\\
	\curvearrowright\,a_\mu=g_{\mu\nu} a^\nu
\end{gather*}

Distance:
\begin{itemize}
	\item $a_\mu a^\mu>0$: spacelike
	\item $a_\mu a^\mu=0$: lightlike
	\item $a_\mu a^\mu<0$: timelike
\end{itemize}

Proper time ($\vec u=\frac{\mathrm{d}\vec l}{\mathrm{d}t}$: velocity in an inertial system $\,\equiv\,$ ordinary velocity):
\[
	\mathrm{d}\tau=\sqrt{1-\frac{u^2}{c^2}}
\]

Proper velocity:
\begin{gather*}
	\vec\eta=\frac{\mathrm{d}\vec l}{\mathrm{d}\tau}=\frac{1}{\sqrt{1-\frac{u^2}{c^2}}}\,\vec u\\
	\eta^\mu=\frac{\mathrm{d}x^\mu}{\mathrm{d}\tau};\,\,\,\eta'^\mu=\Lambda^\mu_{\,\nu}\eta^\nu
\end{gather*}

Relativistic momentum:
\[
	\vec p=m\eta=\frac{m\vec u}{\sqrt{1-\frac{u^2}{c^2}}}
\]

Relativistic energy:
\[
	E=\frac{mc^2}{\sqrt{1-\frac{u^2}{c^2}}}\quad\curvearrowright\,p^0=\frac{E}{c};\,p^i=\vec p^i
\]

Rest and kinetic energy:
\[
	E_\mathrm{rest}=mc^2;\quad E_\mathrm{kin}=E^2-mc^2
\]

Relativistic energy:
\begin{gather*}
	p\cdot p=p_\mu p^\mu=-(p^0)^2+\vec p^2=-m^2c^2\\
	\curvearrowright\,E^2=m^2 c^4+p^2 c^2
\end{gather*}

Second rank tensor:
\begin{gather*}
	t'^{\mu\nu}=\Lambda^\mu_{\,\lambda} \Lambda^\nu_{\,\sigma} t^{\lambda\sigma}\\
	\Lambda=\begin{bmatrix}\gamma & -\gamma\beta & 0 & 0\\ -\gamma\beta & \gamma & 0 & 0\\ 0 & 0 & 1 & 0\\ 0 & 0 & 0 &1\end{bmatrix}\\
		t^{\mu\nu}=\begin{bmatrix}t^{00} & t^{01} & t^{02} & t^{03}\\ t^{10} & t^{11} & t^{12} & t^{13}\\ t^{20} & t^{21} & t^{22} & t^{23}\\ t^{30} & t^{31} & t^{32} & t^{33}\end{bmatrix}
\end{gather*}

(Anti-)symmetric tensor:
\[
	t^{\mu\nu}=(-)t^{\nu\mu}
\]

General antisymmetric tensor:
\[
	\begin{bmatrix}0 & t^{01} & t^{02} & t^{03}\\ -t^{01} & 0 & t^{12} & t^{13}\\ -t^{02} & -t^{12} & 0 & t^{23}\\ -t^{03} & -t^{13} & -t^{23} & 0\end{bmatrix}
\]

Field tensor:
\[
	F^{\mu\nu}=\begin{bmatrix}0 & \nicefrac{E_x}{c} & \nicefrac{E_y}{c} & \nicefrac{E_z}{c}\\ -\nicefrac{E_x}{c} & 0 & B_z & -B_y\\ -\nicefrac{E_y}{c} & -B_z & 0  & B_x\\ -\nicefrac{E_z}{c} & B_y & -B_x & 0\end{bmatrix}
\]

Dual tensor: ($\frac{\vec E}{c}\to\vec B,\,\,\vec B\to-\frac{\vec E}{c}$):
\[
	G^{\mu\nu}=\begin{bmatrix}0 & B_x & B_y & B_z\\ -B_x & 0 & -\nicefrac{E_z}{c} & \nicefrac{E_y}{c}\\ -B_y & \nicefrac{E_z}{c} & 0 & -\nicefrac{E_x}{c}\\ -B_z & -\nicefrac{E_y}{c} & \nicefrac{E_x}{c} & 0\end{bmatrix}
\]

Current density 4-vector:
\[
	j^\mu=(c\varrho,\,j_x,\,j_y,\,j_z)^t
\]

Continuity equation:
\[
	\frac{\partial j^\mu}{\partial x^\mu}=0
\]

Maxwell's equations (summation over $\nu$):
\[
	\frac{\partial F^{\mu\nu}}{\partial x^\nu}=\mu_0 j^\mu\quad\frac{\partial G^{\mu\nu}}{\partial x^\nu}=0
\]

Vector potential of the magnetic field ($\vec\nabla\cdot\vec B=0$):
\begin{gather*}
	\vec B=\vec\nabla\wedge\vec A\\
	\curvearrowright\,\vec\nabla\wedge\vec E=-\frac{\partial}{\partial t}\,\vec B=-\frac{\partial}{\partial t}\,\left(\vec\nabla\wedge\vec A\right)\\
	\curvearrowright\,\vec\nabla\wedge\left(\vec E+\frac{\partial\vec A}{\partial t}\right)=0\\
	\curvearrowright\,\exists\phi:\,\vec\nabla\phi=-\left(\vec E+\frac{\partial\vec A}{\partial t}\right)\\
	\curvearrowright\,\vec E=-\vec\nabla\phi-\frac{\partial\vec A}{\partial t}
\end{gather*}

\begin{gather*}
	\vec\nabla\cdot\vec E=\vec\nabla\cdot\left(-\vec\nabla\phi-\frac{\partial\vec A}{\partial t}\right)=-\vec\nabla^2\phi-\frac{\partial}{\partial t}\,\left(\vec\nabla\cdot\vec A\right)=\frac{1}{\varepsilon_0}\varrho\\
	\curvearrowright\,\vec\nabla^2\phi+\frac{\partial}{\partial t}\,\left(\vec\nabla\cdot\vec a\right)=-\frac{1}{\varepsilon_0}\varrho
\end{gather*}

\begin{gather*}
	\vec\nabla\wedge\vec B=\vec\nabla\wedge\left(\vec\nabla\wedge\vec A\right)=\mu_0\vec j+\mu_0\varepsilon_0\frac{\partial\vec E}{\partial t}=\mu_0\vec j-\mu_0\varepsilon_0\vec\nabla\left(\frac{\partial\phi}{\partial t}\right)-\mu_0\varepsilon_0\frac{\partial^2\vec A}{\partial t^2}\\
	\curvearrowright\,\left(\vec\nabla^2\vec A-\mu_0\varepsilon_0\frac{\partial^2\vec A}{\partial t^2}\right)-\vec\nabla\left(\vec\nabla\cdot\vec A+\mu_0\varepsilon_0\frac{\partial\phi}{\partial t}\right)=-\mu_0\vec j
\end{gather*}
\end{document}
